\Titre{Counter-Measures\&Attacks}

\begin{document}

\begin{reveals}
		
\maketitle



\begin{comment}
  National  Treasure[70]:
Historian  and  treasure  hunter  Benjamin  FranklinGates (Cage) races to find the legendary Templar Treasure before a team ofmercenaries.
%
–Steganography: %
The premise of the film is that on the back of the Declaration of
Independence, signed on July 4, 1776, by 55 Founding Fathers (at least
nine of whom were Freemasons), hides a coded, unseen map that points
to the secret location of the fabled and massive Templar Treasure that
was discovered by the Knights Templar and later protected by the
Freemasons. This film is a real cybersecurity goldmine, covering a
large number of cybersecurity notions in addition to steganography, so
let me provide a detailed description.Please, try to stay with me as
the plot thickens.
%
–Multiple-stage attack, Masquerading attack, Forging, Pseudonym,
Social engineering, Invisible ink, Hacking, Sensor attack, Biometrics,
Integrity, Password guessing and Background knowledge:
%
In order to protect the Declaration of Independence from the
mercenaries and read the hidden message, Gates decides to steal the
Declaration himself. To that end, he carries out a multiple-stage
attack with the help of his sidekick Riley Poole.
%
First, Gates,
pretending to be a tourist shooting some photos while visiting the
NationalArchives, takes a picture of an ID badge of one of the
Archive’s custodians and then creates a fake ID with his own face.
% stage 0, social engineering before password guessing
When he visited Chase in her office, introducing himself with the
pseudonym Paul Brown, Gates noticed that she was missing this button
from her collection, so he knows that she will open the giftbox he
sent her and touch the button to put it in her display case.
% first stage of the password-guessing attack
He then smears a rare George Washington’s campaign button with
invisible ink and sends it to Abigail Chase, an archivist at the
National Archives whom he had tried to warn about the impending
threat. 
% second stage of password guessing
Poole, in the meantime, has hacked into the network of the Archives and
uses a laser hidden in a camera to attack the sensor of the cage that
protects the Declaration to raise the temperature over the
threshold. The archivists therefore remove the Declaration from display
and take it to the Preservation Room for tests.
% third stage for password guessing
Chase goes to the Room too for a check and types the password on a
keyboard, thus unknowingly smearing the keys with the invisible ink
that was on the button that Gates sent her.
% masquerade x 2
Later that evening, there is a gala at the Archives. Gates masquerades
as a custodian using the fake ID badge and, once inside, changes
clothes into a Black Tie, the semi-formal Western dress code for
evening events, so as to mingle with the other guests.
% Masquerading\& Biometrics
He briefly meets Chase and, with an excuse, takes a glass that she has
been drinking from, from which, using a chemical reagent, he
reconstructs her fingerprints and puts them on a glove, which he uses
to open a door controlled by the fingerprint reader 
% Biometrics trumping, the hard way
(while the mercenaries, who are also trying to get to thePreservation
Room, knock a security guard out and drag him to another reader to
open the door with the guard’s actual finger).
% Replay attack, inauthentic, also unavailability of the real camera
Gates needs now to walk through a corridor, so Poole turns off the
surveillance cameras and replaces the feed with an earlier one, so that
the security guards watching the monitors see only an empty
corridor.
%
There is one final biometric access control device left for Gates: the
keyboard that Chase used earlier that day.
% password guessing: generator or background knowledge
He shines a blacklight lamp on the keys to see that the ones smeared
with the invisible ink are: A E F G L O R V Y. Poole uses an automatic
anagram generator to create a list of possible anagrams, but Gates uses
backgroundknowledge about Chase, an historian herself, to correctly
guess that Chase pressed E and L twice and the password is “Valley
forge”, the location where the American Continental Army made camp
during the winter of 1777–1778 and which is nowadays often called the
birthplace of the American Army.
% Attack target. 
Gates can now enter the Preservation Room and take the Declaration.
% Exercice: Reconstruct attack path/tree from this scenario and a few
% (brutal) alternatives

%
- Indistinguishability, Swapping and Tracing: 
%
With the Declaration of Independence folded and tucked under his arm,
Gates has to go through the shop, which sells actual-size reproductions
of the Declaration. 
% mis-identification
A cashier, seeing the real Declaration tucked under Gates’ arm, thinks
that he is trying to steal one of the reproductions, so he is forced
to pay for it.
% indistinguishability
What we the viewers don’t know yet, but will soon find out, is that he
actually pays both for the real Declaration and for a reproduction, so
that when the bad guys catch up with him, he swaps the two
declarations and gives them the replica without them noticing at
first, and he can getaway.
% tracing, audit
“I thought it would be a good idea tohave a duplicate. It
turned out I was right.” Unfortunately, in the shop he did not have
enough cash, so he paid with his credit card, which means that the FBI
can trace the payment and find out that he is not Paul Brown
but discover his real identity and where he lives.
%
–Ottendorf cipher: 
% Again multiple-stage attack, this time relying on a cipher,
% and how to break it given its structure. But also counter-measures
% defeated to access 
As a consequence, Gates and Poole cannot use the clean-room
environment that they had set up, so, together with Gates’ father
and with Chase, who in the meantime has semi-willingly joined them,
they use more artisanal methods to reveal the message hidden on the
back of theDeclaration.
%
Using lemon juice and their own breaths as heat sources, Gatesand
Chase expose the message, which is constituted by a series of triples
of numbers. It is an Ottendorf cipher and the following dialogue takes
place to provide a useful explanation for the non-expert viewers:
Poole: Will somebody please explain to me what these magic numbers
are?%
Chase:It’s an Ottendorf cipher.%
Gates’ father: That’s right.%
Poole:Oh, OK. What’s an Ottendorf cipher?%
Gates’ father: They’re just codes.%
Gates: Each of these three numbers corresponds to a word in a key.%
Chase: Usually a random book or a newspaper article.%
Gates: In this case, the Silence Dogood letters. So it’s like[Pointing
to the numbers in a triple, ed]the page number of the key text, the
line on the page, and the letter in that line.%
%
As mentioned earlier in the film, Silence Dogood was the pseudonym
that 16 year old Benjamin Franklin used to get his letters published
in the “New-England Courant”, a newspaper founded and published by his
brother James Franklin. Gates and his allies use the cipher (and some
additional knowledgethey have on the history of the Liberty Bell and
on the design of the \$100banknote) to find the hidden message in the
letters.
% multiple-stage/security by obscurity
This proves to be another clue that allows them to find a special pair
of spectacles invented by BenjaminFranklin, which in turn allows them
to read another invisible message on the back of the Declaration of
Independence, which eventually, after some additional hunting, allows
them to find the location of the treasure.
%
Interestingly, the film also contains a reference to the famousMonty
Hall problem that originates from the game show “Let’s Make a Deal”
[61]
\end{comment}

\begin{frame}
  \frametitle{Historical background}

  \begin{block}{Before the  1970s}
    \begin{itemize}
    \item Computers only in physically secured environments
    \item Computers rebooted between each computation
    \end{itemize}
  \end{block}

  \pause
  
  \begin{block}{Origin of security concerns}
    \begin{itemize}
    \item Time-sharing: organisations with different needs use a same
      computer for economical purposes (Multics, Unix)
    \item Security goal: Time-sharing should be as secure as non-Time-sharing systems
    \end{itemize}
  \end{block}

  \pause

  \begin{block}{Examples}
    \begin{itemize}
    \item University and DoE/DoD use the same super-computer for simulation
    \item Nowadays: exactly the same concern:
      \begin{itemize}
      \item Avionics: RTCA DO-255
      \item Cloud computing
      \end{itemize}
    \end{itemize}
  \end{block}

\end{frame}

\begin{frame}
  \frametitle{Separation}

  \begin{block}{Separation informal statement}
    The execution of one program cannot influence in any way the
    execution of another program
  \end{block}

  \pause

  \begin{block}{Extensive definition}
    \begin{itemize}[<+->]
    \item A program in an infinite loop cannot preempt the processor
    \item A program with a memory leak cannot hold all the available memory
    \item In TS OS, the processor state has to be cleaned at each
      process change
    \end{itemize}
  \end{block}
\end{frame}

\begin{frame}
  \frametitle{Trusted Computing Base (1/2)}

  \begin{block}{Assume/Guarantee}
    When adding a new component to a system
    \begin{itemize}
    \item Except at the lowest level, use of other components based on assumptions  
      \begin{itemize}
      \item OS on the hardware
      \item Applications/libraries on the OS and other applications/libraries
      \item $\ldots$
      \end{itemize}
    \item The new component guarantees (at some level of certainty) some properties:
      \begin{itemize}
      \item no memory leaks,
      \item bounds on execution time,
      \item \ldots
      \end{itemize}
    \end{itemize}
  \end{block}
\end{frame}

\begin{frame}
  \frametitle{Trusted Computing Base (2/2)}

  \begin{block}{Security in CS}
    \begin{itemize}
    \item Formalisms to state and prove the guarantees provided by a
      component based on some assumptions on other components
    \item Trusted computing base: parts of a system that are not
      analysed and just assumed to provide the stated guarantees
    \item Example: processor state reset:
      \begin{itemize}
      \item register reset not in early version of Unix (unused registers were left as is)
      \item cache never resetted after a context change (leading to
        lot of security attacks, \textit{e.g.} Spectre)
      \end{itemize}
    \item problem compounded in current processors by pipeline,
      speculative execution, and on-the-fly code optimisation
    \end{itemize}
  \end{block}
\end{frame}


\begin{frame}
  \frametitle{Security Solution: Static Configuration}
  
  \begin{block}{Principle}
    Ressources are statically allocated to consumers
  \end{block}

  \begin{block} {Security implications: Access Control}
    \begin{itemize}[<+->]
    \item Access control policy: definition of the possible allocation
      of resources to consumers
    \item Must be defined before the start of the machine
    \item Cannot be changed while it is running
      \begin{center}
        \emph{Static AC}
      \end{center}
    \item Example: bounds on execution time, memory available, etc.
    \item Unmutable policy:
      \begin{center}
        \emph{Mandatory Access Control}
      \end{center}
    \item Allows for a thorough analysis of a system
    \end{itemize}
  \end{block}

\end{frame}


\begin{frame}
  \frametitle{Time separation}

  \begin{block}{Statement}
    The AC policy must specify the computation time allocated to each
    program
  \end{block}

  \begin{itemize}
  \item Hard real-time systems
  \item Allocates a fraction of the processor cycles to each process
  \item In practice:
    \begin{itemize}
    \item implemented in embedded and mission-critical systems for
      safety
      \begin{center}
        the ABS functionality cannot be affected when you start a new
        Ariana Grande song
      \end{center}
    \end{itemize}
  \end{itemize}

\end{frame}


\begin{frame}
  \frametitle{Spatial Separation}

  \begin{block}{Statement}
    Memory available to each application has to be allocated
    statically and be disjoint
  \end{block}

  \begin{itemize}
  \item \textit{a.k.a.} \emph{partitioning}
  \item Implies no direct communication between applications (IPC
    system V), communication through the OS still possible
  \item In practice
    \begin{itemize}
    \item RAM,HDD, SDD are sets of pages, and that set is partitioned,
      and each application receives one subset
    \item No malloc/mmap, or a secure variant
    \item Bounded stack
    \item No pointers unless the Processor/OS guarantees that pointers
      never access a wrong page
    \end{itemize}
  \item Very hard in practice
    \begin{itemize}
    \item Need for a specific language (\textit{e.g.} Lustre) or compiler
    \item Registers and cache still have to shared on current processors
    \end{itemize}
  \end{itemize}

\end{frame}

\begin{frame}
  \frametitle{Access Control and Spatial Separation}

  \begin{itemize}[<+->]
  \item TL;DR: very hard, but with some tolerance, achieved by current
    systems like Linux CGroups
  \item Incidentally, development driven by Cloud-hosting companies
  \item Known tolerances:
    \begin{itemize}
    \item shared libraries
    \item bounds on max memory available, but no hard partition
    \item \texttt{root} still can do whatever he wants
    \end{itemize}
  \end{itemize}

\end{frame}

\begin{frame}
  \frametitle{Partitioning in Information Systems}

  \vfill

  \begin{block}{In a computer network}
    \begin{itemize}
    \item Tasks to be performed by a system are allocated to different
      computers based on their sensitivity
    \item The network is partitioned into \emph{zones} for computers
      of similar security concerns
    \item Access Control on inter-zone exchanges (\textit{a.k.a.}
      \emph{firewall}) 
    \item Example:
      \begin{itemize}
      \item \textcolor{blue}{DMZ} at the border between a
        \textcolor{blue}{corporate network} and the
        \textcolor{blue}{outside} (3 zones), 
      \item \textcolor{blue}{Firewalls} to control packets between these zones
      \end{itemize}

    \end{itemize}
  \end{block}

  \vfill

  \begin{block}{Note}
    \begin{itemize}
    \item Outside of critical systems, there's no way to prove that
      all applications are secure
    \item First glimpse at \emph{Layered Security}/\emph{Defence-in-Depth}
    \end{itemize}
  \end{block}
\end{frame}




\begin{frame}
  \frametitle{Good AC systems}
  \framesubtitle{NEAT Security}

  \begin{itemize}[<+->]
  \item \textcolor{red}{N}on-bypassable: impossible to act outside of
    the bounds of the AC system
  \item \textcolor{red}{E}valuatable: possible communications can be
    assessed before deployment
  \item \textcolor{red}{A}lways invoked: no future rights, the AC
    system must always be queried
  \item \textcolor{red}{T}amperproof: the AC system cannot be changed
    while the system is running
  \end{itemize}
  
  \pause

  \begin{itemize}
  \item These concerns are shared with \emph{safe} systems
  \item Possible on systems dedicated to a task, not in general
    (\textit{e.g.}, no new file and no new user)
  \item The CS Security dilemna:
    \begin{itemize}
    \item we know how to secure a system
    \item we know how to have a usable system
    \item we know how to build a cheap system
    \end{itemize}
    but no one has ever (or ever hopes to have) more than 2 out of 3
  \end{itemize}

\end{frame}



%\begin{frame}
%  \frametitle{Disgression cryptographique}


%  \begin{itemize}[<+->]
%  \item Lors du cours de math{\'e}matiques, on a vu des primitives IND-CCA2
%  \item 
%    \begin{block}{Rappel:}
%      On autorise l'attaquant {\`a} conna{\^\i}tre le texte clair d'autant de
%      chiffr{\'e}s qu'il veut, sauf d'un en particulier. L'attaquant ne
%      doit pas pouvoir obtenir un seul bit d'information sur le clair
%      du chiffr{\'e} sp{\'e}cial
%    \end{block}
%  \item Pire cas: l'attaquant peut d{\'e}chiffrer tous les autres messages
%    \begin{center}
%      \emph{Environnement d'ex{\'e}cution d{\'e}grad{\'e}}
%    \end{center}
%  \item Attaque faible: conna{\^\i}tre un seul bit d'information
%    \begin{center}
%      Interaction minimale entre l'attaquant et le texte clair
%    \end{center}
%  \item Application du principe de s{\'e}paration:
%    \begin{itemize}
%    \item Attaquant $=$ une application
%    \item Texte clair $=$ une application
%    \item On ne veut pas qu'il y ait d'interaction entre l'attaquant
%      et le texte clair quelque soit l'environnement d'ex{\'e}cution ( $=$
%      les autres chiffr{\'e}s)
%    \end{itemize}
%  \end{itemize}
%\end{frame}



\end{reveals}
\end{document}
