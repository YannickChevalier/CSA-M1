\Titre{Cryptographie}



  
\usepackage{ifthen}


\def\txthl#1{ \ifthenelse{\lengthtest{#1 pt<0.5pt}}{\top}{\bot} }

\begin{document}

\begin{reveals}
		
\maketitle


\section{Survol des primitives cryptographiques}


\begin{frame}
  \frametitle{Cryptographie \textit{vs.} Contrôle d'accès}

  \vfill

  \begin{block}{Contrôle d'accès}
    \begin{itemize}
    \item une politique
    \item un mécanisme d'application doit mettre en \oe uvre cette
      politique
    \item mécanisme applicable uniquement sur des composants qu'on
      maîtrise
    \end{itemize}
  \end{block}

  \vfill

  \begin{block}{Cryptographie}
    \begin{itemize}
    \item transformations de données
    \item les propriétés obtenues sont indépendantes
      de la localisation de ces données
    \item permet la communication de données en dehors des composants maîtrisés
    \end{itemize}
  \end{block}

  \vfill
\end{frame}




\begin{frame}
  \frametitle{Primitives Cryptographiques}

  \vfill

   \begin{block}{Primitives ?}
     \begin{itemize}
     \item type d'opérations sur les messages/textes
     \item classement en fonction  de leur effet
     \end{itemize}
  \end{block}

  \vfill

  \begin{block}{Grands types de primitives}
    \begin{itemize}
    \item fonctions de hachage (pour la sûreté et la sécurité)
    \item chiffrement symétrique (2 personnes, 1 clef partagée pour le chiffrement et le déchiffrement)
    \item chiffrement asymétrique (1 personne, 2 clefs, une pour le
      chiffrement, une pour le déchiffrement)
    \item signature digitale (1 personne, 2 clefs, une pour la
      signature, une pour la validation des signatures)
    \end{itemize}
  \end{block}

  \vfill

\end{frame}

\begin{frame}
  \frametitle{Base de la Cryptographie}

  \vfill

   \begin{block}{Ingrédients essentiels}
     \begin{itemize}
     \item toutes les primitives robustes nécessitent un générateur
       aléatoire
     \item toutes les primitives robustes partent d'un problème à ``porte dérobée'':
       \begin{itemize}
       \item il est facile d'effectuer un calcul dans un sens
       \item il est très difficile d'inverser ce calcul
       \item sauf si on peut passer par une porte dérobée
       \end{itemize}
     \end{itemize}
  \end{block}

  \vfill

  \begin{block}{Évaluation}
    Pour chaque primitive:
    \begin{itemize}
    \item certaines opérations faciles, généralement implémentées dans
      des librairies
    \item la robustesse est définie par un devinette qui conclut que sans
      porte dérobée, il n'est pas possible de faire mieux que deviner
      au hasard la réponse
    \end{itemize}
  \end{block}

  \vfill

\end{frame}

\section{Chiffrement asymétrique}


\begin{frame}
  \frametitle{Chiffrement asymétrique}

  \vfill
  
  \begin{block}{Contexte}
    \begin{itemize}
    \item découverte au début des années 70
    \item exemple connu historiquement: RSA
    \item chaque personne dispose de deux clefs:
      \begin{itemize}
      \item une clef distribuée à tout le monde, qui sert à chiffrer des messages
      \item une clef gardée secrète, qui permet de déchiffrer des messages
      \end{itemize}
    \end{itemize}
  \end{block}
  
  \vfill

  \begin{block}{Dans la vie de tous les jours}
    laisser un message sur la messagerie d'une personne:
    \begin{itemize}
    \item avec un annuaire, tout le monde peut connaître le numéro de
      téléphone de cette personne et laisser un message
    \item seule la personne ayant le téléphone pourra connaître le
      contenu des messages sur sa messagerie
    \end{itemize}
  \end{block}

  \vfill


\end{frame}


\begin{frame}
  \frametitle{Cycle d'utilisation: Initialisation}

  \vfill

  \begin{block}{Procédure}
    \begin{itemize}
    \item un paramètre de difficulté (un nombre de bits)
    \item une personne \(A\) crée un couple \((KA,KA^{-1})\) de clefs
      à partir d'un générateur aléatoire
    \item pour simplifier, \(KA\) est distribuée à tout le monde, et
      \(KA^{-1}\) reste connue de \(A\) seul
    \end{itemize}
  \end{block}

  \vfill

  \begin{block}{Risques}
    \begin{itemize}
    \item si le générateur aléatoire marche mal, une personne
      extérieure peut tenter de deviner les clefs qui seront produites
    \item réutilisation de paramètres d'une application à l'autre
    \item réseaux: pour le Web, supporter de vieux navigateurs
      signifie en général accepter des communications avec des
      paramètres trop faibles
    \end{itemize}
  \end{block}

  \vfill

\end{frame}


\begin{frame}
  \frametitle{Cycle d'utilisation: Utilisation}

  \vfill

  \begin{block}{Opérations faciles}
    \begin{itemize}
    \item en connaissant un message \(m\) et une clef publique \(k\),
      il est \textbf{facile} d'obtenir \(c=encp(m,k)\), le chiffré de
      \(m\) par \(k\)
    \item en connaissant un message chiffré \(c=encp(m,k)\) et la clef
      privée correspondante \(k^{-1}\), il est \textbf{facile} de déchiffrer
      \(c\) pour calculer \(m\)
    \end{itemize}
  \end{block}

  \vfill

  \begin{block}{Modèle logique}
    \[
      \left\lbrace
        \begin{array}{r@{\ensuremath{\rightarrow}}l}
          m,k & encp(m,k) \\
          encp(m,k),k^{-1} & m\\
        \end{array}
      \right.
    \]
  \end{block}

  \vfill


\end{frame}

\begin{frame}
  \frametitle{Cycle d'utilisation: robustesse}

  \vfill

   \begin{block}{Jeu de base}
     \begin{enumerate}
     \item le joueur choisit deux messages de même longueur \(m\) et \(m'\)
     \item il reçoit en retour soit \(encp(m,k)\), soit \(encp(m',k)\)
     \item le joueur doit deviner lequel des deux messages il a reçu
     \end{enumerate}
  \end{block}

  \vfill

  \begin{block}{Améliorations}
    \begin{itemize}
    \item à tout moment, le joueur peut demander à déchiffrer
      n'importe quel message qu'il construit tant que ce n'est pas le
      message qu'il a reçu à l'étape 2
    \item la chiffrement est robuste si la probabilité qu'il a deviné
      correctement est \(\frac 12\), quelque soit le message renvoyé
    \end{itemize}
  \end{block}

  \vfill



\end{frame}

\begin{frame}
  \frametitle{Exemple}

  \vfill

   \begin{block}{RSA}
     \begin{enumerate}
     \item publique: un module \(N\), un entier \(k\)
     \item privé: un entier \(k^{-1}\) tel que:
       \[
         \forall 0\le m < N, \left(m^k\right)^{k^{-1}} = m \mod N
       \]
     \item chiffrement: \(encp(m,k)=m^k \mod N\)
     \item déchiffrement \( c^{k^{-1}} \mod N\)
     \end{enumerate}
  \end{block}

  \vfill

  \begin{block}{Note}
    \begin{itemize}
    \item propriété importante: le déchiffrement ``réussit'' toujours,
      mais si on utilise la mauvaise clef on obtient un message sans signification
    \item pirmitive pas robuste (pourquoi ?)
    \end{itemize}
  \end{block}

  \vfill
  
  \begin{block}{Moralité}
    \begin{itemize}
    \item lorsqu'on chiffre deux fois un même message, on doit obtenir
      deux résultats indépendants
    \item on a aussi besoin d'aléatoire pour le chiffrement
    \end{itemize}
  \end{block}

  \vfill

\end{frame}

\section{Signature Digitale}




\begin{frame}
  \frametitle{Signature Digitale}

  \vfill
  
  \begin{block}{Contexte}
    \begin{itemize}
    \item historiquement basée sur le chiffrement asymétrique 
    \item exemple connu historiquement: DSA (signature RSA)
    \item chaque personne dispose de deux clefs:
      \begin{itemize}
      \item une clef distribuée à tout le monde, qui sert à valider des signatures
      \item une clef gardée secrète, qui permet de signer des messages
      \end{itemize}
    \end{itemize}
  \end{block}
  
  \vfill

  \begin{block}{Dans la vie de tous les jours}
    affichage du numéro d'un correspondant:
    \begin{itemize}
    \item avec un annuaire/liste de contact, tout le monde peut
      connaître la personne ayant le numéro appelant
    \item seule la personne ayant le téléphone pourra appeler avec ce
      numéro
    \item ne pas montrer son numéro revient à ne pas signer son coup de fil
    \end{itemize}
  \end{block}

  \vfill


\end{frame}


\begin{frame}
  \frametitle{Cycle d'utilisation: Initialisation}

  \vfill

  \begin{block}{Procédure (idem chiffrement}
    \begin{itemize}
    \item un paramètre de difficulté (un nombre de bits)
    \item une personne \(A\) crée un couple \((KA,KA^{-1})\) de clefs
      à partir d'un générateur aléatoire
    \item pour simplifier, \(KA\) est distribuée à tout le monde, et
      \(KA^{-1}\) reste connue de \(A\) seul
    \end{itemize}
  \end{block}

  \vfill

  \begin{block}{Risques}
    \begin{itemize}
    \item idem chiffrement
    \end{itemize}
  \end{block}

  \vfill

\end{frame}


\begin{frame}
  \frametitle{Cycle d'utilisation: Utilisation}

  \vfill

  \begin{block}{Opérations faciles}
    \begin{itemize}
    \item en connaissant un message \(m\) et la clef de signature
      publique \(k^{-1}\), il est \textbf{facile} d'obtenir
      \(c=sigp(m,k^{-1})\), la signature digitale de \(m\) par
      \(k^{-1}\)
    \item en connaissant un message signé \(s=sigp(m,k^{-1})\) et la
      clef de validation correspondante \(k\), il est \textbf{facile} de
      vérifier que \(s\) est la signature de \(m\)
    \end{itemize}
  \end{block}

  \vfill

  \begin{block}{Modèle logique}
    \[
      \left\lbrace
        \begin{array}{r@{\ensuremath{\rightarrow}}l}
          m,k^{-1} & sigp(m,k^{-1}) \\
          m,sigp(m,k^{-1}),k & \top\\
        \end{array}
      \right.
    \]
  \end{block}

  \vfill


\end{frame}

\begin{frame}
  \frametitle{Cycle d'utilisation: robustesse}

  \vfill

   \begin{block}{Jeu de base}
     \begin{enumerate}
     \item le joueur demande la signature de messages de son choix
     \item le joueur gagne s'il peut produire un couple \((m,\sigma)\)
       où \(\sigma=sigp(m,k^{-1})\) 
     \end{enumerate}
  \end{block}

  \vfill

  \begin{block}{Améliorations}
    \begin{itemize}
    \item la signature est robuste si la probabilité de produire un
      couple correct est négligeable (zéro)
    \item (multiplication) utiliser la même mécanisme que pur RSA
      n'est pas très robuste
    \item on utilise en général une fonction de hachage pour rendre la
      signature robuste
    \end{itemize}
  \end{block}

  \vfill



\end{frame}


\section{Fonction de hachage}




\begin{frame}
  \frametitle{Fonction de hachage}

  \vfill
  
  \begin{block}{Contexte}
    \begin{itemize}
    \item vient des codes correcteurs d'erreur
    \item exemples connus: SHA1, MD5 
      \begin{itemize}
      \item une fonction qui calcule un message de taille fixée en
        fonction d'un message de taille quelconque
      \end{itemize}
    \end{itemize}
  \end{block}
  
  \vfill

  \begin{block}{Dans la vie de tous les jours}
    enregistrer ses contacts avec un surnom:
    \begin{itemize}
    \item peu de chances de connaître 2 personnes différentes ayant
      les mêmes nom etprénoms
    \item sans la liste de contacts, impossible de contacter une
      personne à partir du surnom
    \item si on choisit les surnoms au hasard, peu de chances que
      quelqu'un d'autre puisse trouver un des surnoms utilisés
    \end{itemize}
  \end{block}

  \vfill


\end{frame}




\begin{frame}
  \frametitle{Cycle d'utilisation: Utilisation}

  \vfill

  \begin{block}{Opérations faciles}
    \begin{itemize}
    \item en connaissant un message \(m\) et la fonction \(h(\cdot)\),
      il est \textbf{facile} d'obtenir \(c=h(m)\), le haché de \(m\)
    \item en connaissant un message haché \(b=h(m)\) et le message
      original \(m\), il est \textbf{facile} de vérifier si \(b\) est
      le haché de \(m\)
    \end{itemize}
  \end{block}

  \vfill

  \begin{block}{Modèle logique}
    \[
      \left\lbrace
        \begin{array}{r@{\ensuremath{\rightarrow}}l}
          m & h(m)\\
        \end{array}
      \right.
    \]
  \end{block}

  \vfill


\end{frame}


\begin{frame}
  \frametitle{Cycle d'utilisation: robustesse}

  \vfill

   \begin{block}{Recherche d'antécédents}
     du plus faible au plus dangereux
       \begin{itemize}
       \item il est possible de calculer \(x,y\) tels que \(h(x) = h(y)\)
       \item connaissant \(x,h(x)\), il est possible de calculer \(y\) tel que
         \(h(x) = h(y)\)
       \end{itemize}
  \end{block}

  \vfill

  \begin{block}{Notes}
    \begin{itemize}
    \item Pour les fonctions de hachage communément utilisées, pas de
      critère basé sur des jeux
    \item (taille des espaces entrée/sortie): il y a toujours une
      infinité de collisions, mais selon la fonction il sera plus ou
      moins dur d'en calculer
    \end{itemize}
  \end{block}

  \vfill

\end{frame}




\begin{frame}
  \frametitle{Variante: HMAC}

  \vfill

  \begin{center}
    HMAC = hash \(+\) nombre aléatoire partagé
  \end{center}
  \vfill

  \begin{block}{HMAC et signature digitale}
    \begin{itemize}
    \item connaissant \(m\), la valeur \(h(m,r)\) peut être produite
      par toute personne connaissant \(r\)
    \item connaissant \(m,c\), toute personne connaissant \(r\) peut
      vérifier que \(c=h(m,r)\)
    \item signature light, symétrique, propre à un groupe de sujets
    \item avantage: très rapide à calculer
    \end{itemize}
  \end{block}

  \vfill

  \begin{block}{Signature digitale et h}
    \begin{itemize}
    \item RSA avec application de la clef privée sur le haché de \(m\)
      est très sûr
    \item en plus les temps de calcul sont plus courts
    \end{itemize}
  \end{block}

  \vfill



\end{frame}



\section{Chiffrement symétrique}




\begin{frame}
  \frametitle{Chiffrement symétrique}

  \vfill
  
  \begin{block}{Contexte}
    \begin{itemize}
    \item historiquement le seul chiffrement jusque dans les années 70
    \item chiffrement de César, de Vigenère, Enigma,
    \item notion de chiffrement parfait, mais pas de jeu difficile
      pour les chiffrements en pratique
    \end{itemize}
  \end{block}
  
  \vfill

  \begin{block}{Chiffrement parfait}
    \begin{itemize}
    \item Vernam: un générateur aléatoire parfait \(k\)
    \item chiffrement: \(m \oplus k\) (ou exclusif bit à bit)
    \item déchiffrement \((m \oplus k)\oplus k = m\) 
    \end{itemize}
  \end{block}

  \vfill


\end{frame}


\begin{frame}
  \frametitle{Cycle d'utilisation: initialisation}

  \vfill

  \begin{block}{Entente}
    \begin{itemize}
    \item 2 personnes souhaitant communiquer s'entendent sur un nombre
      aléatoire (contexte de sécurité ou clef secrète)
    \item le contexte de sécurité est utilisé pour initialiser un
      générateur de nombre aléatoires (puis ou exclusif avec les nombres générés)
    \item la clef est utilisée pour chiffrer bloc par bloc un message
      avec chaînage entre les différents blocs
    \end{itemize}
  \end{block}

  \vfill


\end{frame}



\begin{frame}
  \frametitle{Cycle d'utilisation: Utilisation}

  \vfill

  \begin{block}{Opérations faciles}
    \begin{itemize}
    \item en connaissant un message \(m\) et la clef \(k\), il est
      \textbf{facile} d'obtenir \(c=encs(m,k)\), le chiffré de \(m\)
    \item en connaissant un message chiffré \(c=encs(m,k)\) et la clef
      \(k\), il est \textbf{facile} de calculer \(m\)
    \end{itemize}
  \end{block}

  \vfill

  \begin{block}{Modèle logique}
    \[
      \left\lbrace
        \begin{array}{r@{\ensuremath{\rightarrow}}l}
          m,k & encs(m,k)\\
          encs(m,k),k & m\\
        \end{array}
      \right.
    \]
  \end{block}

  \vfill


\end{frame}


\begin{frame}
  \frametitle{Cycle d'utilisation: robustesse}

  \vfill

  \begin{block}{Jeux cryptographiques}
     \begin{itemize}
     \item des jeux peuvent être définis comme pour le chiffrement asymétrique
     \item mais les chiffrements utilisés en pratique ne sont pas montrés sûrs
     \end{itemize}
  \end{block}

  \vfill

  \begin{block}{Notes}
    \begin{itemize}
    \item le chiffrement par blocs est déterministe (ex: AES), donc
      moins sécurisé
    \item pour le rendre moins déterministe, on utilise en plus le bloc
      chiffré précédent pour calculer le chiffré du bloc courant
    \item vecteur d'initialisation: valeur utilisée pour chiffré le premier bloc
    \end{itemize}
  \end{block}

  \vfill

\end{frame}






\end{reveals}
\end{document}
 
