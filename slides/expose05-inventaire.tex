\Titre{Inventaire des composants}



  
\usepackage{ifthen}



\begin{document}

\begin{reveals}
		
  \maketitle


  \begin{frame}
    \frametitle{Retour sur les composants}

  \vfill
  \begin{block}{Composants concrets}
    \begin{itemize}
    \item l'environnement physique (sites, batiments, salles, badges,\(\ldots\))
    \item l'environnement SI déployés (serveurs, clients, routeurs, \(\ldots\)
    \item l'environnement SI réel (téléphones portables, tablettes, portables,\(\ldots\))
    \item les personnes recensées et les visiteurs
    \end{itemize}
  \end{block}

  \vfill

\begin{block}{Ce cours}
  \begin{itemize}
  \item recensement systématique des composants
  \item avec un focus sur la sécurité dans les différents cas
  \end{itemize}
  \end{block}

  \vfill

\end{frame}

\begin{frame}
  \frametitle{Identification des composants}

  \vfill

   \begin{block}{Actifs primordiaux}
     \begin{itemize}
     \item données
     \item processus métiers (services propres à l'entreprise)
     \end{itemize}
  \end{block}

  \vfill

\begin{block}{Éléments supports}
  \begin{itemize}
  \item applications (mail, serveur SMB/NFS)
  \item système(s) d'exploitation
  \end{itemize}
  \end{block}

  \vfill

\begin{block}{Équipements}
  \begin{itemize}
  \item biens propres: commutateur, routeur, serveurs, ordinateurs de l'entreprise
  \item biens des usagers: tablettes, smartphones
  \end{itemize}
  \end{block}

  \vfill



\end{frame}


\begin{frame}
  \frametitle{Inventaire des données}

  \vfill

   \begin{block}{Identification des données sensibles}
     \begin{itemize}
     \item pour la sécurité: mots de passe, badge, \(\ldots\)
     \item pour l'entreprise: plans marketing, stratégie industrielle,
       fichiers clients, contrats, \(\ldots\)
     \end{itemize}
  \end{block}

  \vfill

\begin{block}{Processus métiers}
  \begin{itemize}
  \item il est important que certaines procédures restent secrètes
  \item dans l'informatique, l'architecture d'une application peut
    être le c\oe ur d'une entreprise
  \end{itemize}
  \end{block}

  \vfill

\end{frame}

\begin{frame}
  \frametitle{Inventaire des biens}

  \vfill

   \begin{block}{Quelques outils}
     \begin{itemize}
     \item Identification des ordinateurs connectés au réseau:
       ServiceNow, HP OpenView, arp-scan, nmap
     \item Liste des programmes sur les ordinateurs: AIDA64 (Windows,
       MacOS, iOS, Android), dpkg, find (Ubuntu, Linux)
     \end{itemize}
  \end{block}

  \vfill
\begin{block}{Problème pratique}
  \begin{itemize}
  \item idéal: l'administrateur décide des machines et logiciels installés sur le réseau
  \item pratique: les utilisateurs installent des logiciels 
  \item résolution: focus sur les parties maîtrisées, et séparation dans le réseau
    basée sur les parties non-maîtrisées
  \end{itemize}
  \end{block}

  \vfill
\end{frame}

\section{Inventaire et sécurisation du réseau}


\begin{frame}
  \frametitle{Maîtrise du Réseau}

  \vfill

   \begin{block}{Zones indépendantes}
     \begin{itemize}
     \item zone: partie du réseau sécurisée par un pare-feu (entrant et sortant)
     \item séparation suivant des considérations diverses:
       \begin{itemize}
       \item menace: public, visiteurs physiques, type d'emploi
       \item biens à protéger: réseau par projet ou type de données
         (comptes de l'entreprise \textit{vs} recherche)
       \item importance du bien: \textbf{administration système et
           sécurité seulement à partir de certains ordinateurs}
       \end{itemize}
     \end{itemize}
  \end{block}

  \vfill

  \begin{block}{Authentification diverse}
    \begin{itemize}
    \item méthode(s) d'authentification varient suivant la zone
    \item authentification à distance (Radius, Kerberos) par des
      serveurs protégés
    \end{itemize}
  \end{block}

  \vfill

\end{frame}

\begin{frame}
  \frametitle{BYOD}
  \framesubtitle{Bring Your Own Device}

  \vfill

   \begin{block}{Définition}
     \begin{itemize}
     \item le fait, par les utilisateurs, d'apporter leurs ordinateurs
       au bureau
     \item pratique interdite il y a quelques années pour raisons de sécurité
     \item smartphone, tablettes, etc: il faut s'en accomoder
     \end{itemize}
  \end{block}

  \vfill

\begin{block}{En pratique}
  \begin{itemize}
  \item standard: séparation dans une zone spécifique (e.g., eduroam)
  \item sécurité haute: besoin d'authentification cryptographique du
    matériel connecté dans les zones sécurisées (MAC insuffisant)
  \item vérification des logiciels installés possible mais lourde
    (droits administrateurs sur machines privée)
  \end{itemize}
  \end{block}

  \vfill

\end{frame}

\begin{frame}
  \frametitle{Contrôle des échanges entre zones}

  \vfill
  \begin{block}{White/Black listing}
    \begin{itemize}
    \item white listing: n'autoriser que les (flux d'information) recensés et autorisés
    \item black listing: n'interdire que les (flux d'information) recensés et interdits
    \item white listing plus exigeant, mais apporte un meilleur niveau de sécurité
    \end{itemize}
  \end{block}

  \vfill


   \begin{block}{Inventaire des flux}
     \begin{itemize}
     \item lister les connexions utilisées entre les différents zones
       (nmap, inventaire des processus métier, sondages des personnes)
     \item pare-feux rejetant tous les flux sauf ceux autorisés
     \end{itemize}
  \end{block}

  \vfill
\begin{block}{Contournement}
  \begin{itemize}
  \item un des motifs d'introduction des applications Web: passer par
    le port 80/443 qui est en général ouvert
  \item déplacement d'un ordinateur d'une zone vers une autre
  \end{itemize}
  \end{block}

  \vfill
\end{frame}

\begin{frame}
  \frametitle{Cas particulier: Internet}

  \vfill

   \begin{block}{Zone démilitarisée (DMZ}
     \begin{itemize}
     \item intégrité moyenne, confidentialité basse
     \item zone à l'intérieur de laquelle on contrôle tous les accès
     \item filtrage des paquets provenant d'Internet plus contrôlé
     \item limite sévère des connexions entrantes: quelques ports utilisés
     \item mise place de \textbf{systèmes de détection d'intrusion}
       (IDS) et de \textbf{systèmes de protection contre les
         intrusions} (IPS)
     \end{itemize}
  \end{block}

  \vfill

  \begin{block}{Contrôle des flux venant du SI}
    \begin{itemize}
    \item nécessaire de limiter les fuites d'information (fichier mis en téléchargement)
    \item nécessaire de protéger aussi l'intégrité de la DMZ (mise en
      ligne de programmes pouvant servir de relais vers d'autres
      parties du SI)
    \end{itemize}
  \end{block}

  \vfill
\end{frame}


\begin{frame}
  \frametitle{Accès à distance}

  \vfill

   \begin{block}{Cas d'utilisation}
     \begin{itemize}
     \item remontées en temps-réel du terrain
     \item télétravail
     \item administration à distance
     \end{itemize}
  \end{block}

  \vfill

\begin{block}{Recommandations}
  \begin{itemize}
  \item serveurs d'authentification (Kerberos, Radius)
  \item concentrateur VPN (remplacement IP par IPSec),ssh (au-dessus de TCP)
  \item serveur application dédié (Remote Access Server) au transfert authentifié des données
  \item dans tous les cas, le matériel utilisé par le client
    \textbf{doit} être fourni et géré par l'entreprise (pas de BYOD)
  \end{itemize}
  \end{block}

  \vfill

\end{frame}

\begin{frame}
  \frametitle{Mail}

  \vfill

   \begin{block}{POP, SMTP, IMAP}
     \begin{itemize}
     \item pop, imap: protocole de synchronisation de boîtes mail
     \item pop marche par copies locales, imap par accès à distance à
       une unique boîte mail
     \item smtp: protocole d'envoi de mails
     \end{itemize}
  \end{block}

  \vfill

\begin{block}{Authentification (POP,IMAP)}
  \begin{itemize}
  \item ssh + mot de passe pour l'authentification du client (et accès à distance)
  \item BYOD: les mots de passe des utilisateurs se retrouvent un peu
    partout (Google, Outlook, mais aussi des applications ``sympas'')
  \end{itemize}
  \end{block}

  \vfill
\begin{block}{Authentification par externe (SMTP)}
  \begin{itemize}
  \item pour l'utilisation d'un mot de passe, il faut accepter un
    niveau de protection plus faible par Google (car ils n'ont pas
    confiance en votre entreprise)
  \item alternative: enregistrer une clef publique d'authentification
    auprès de Google
  \end{itemize}
  \end{block}

  \vfill




\end{frame}



\section{Sécurisation des postes}

\begin{frame}
  \frametitle{Limite (BYOD)}

  \vfill

  \begin{block}{Exclusion du BYOD}
    \begin{itemize}
    \item impossible en pratique d'empêcher les utilisateurs d'amener
      leur téléphone portable
    \item il convient de fournir un réseau Wifi bien séparé du reste
      du réseau de l'entreprise (impression, partage de documents)
    \item mais le mail permet souvent un échange de documents
      incontrôlé (hors log)
    \end{itemize}
  \end{block}

  \vfill

\begin{block}{Suite:}
  \begin{itemize}
  \item focus sur les postes gérés par l'administrateur système
  \item recensement de bonnes pratiques
  \end{itemize}
  \end{block}

  \vfill

\end{frame}

\begin{frame}
  \frametitle{Déploiement de nouveaux logiciels}

  \vfill

   \begin{block}{Par les utilisateurs lambda}
     \begin{itemize}
     \item installer un logiciel, c'est donner les pleins pouvoirs
       (exception: android) à l'auteur sur son ordinateur
     \item extrèmement important de vérifier l'auteur (automatique
       sous Windows et apt/rpm) du logiciel
     \item il faut faire confiance aux procédures des fournisseurs
       (signature MS, Google Play, iTunes) pour vérifier que les
       logiciels distribués ne contiennent pas de virus
     \item note: faire un scan de son ordinateur à partir d'une page
       web fait courir les mêmes risques
     \end{itemize}
  \end{block}

  \vfill

  \begin{center}
    \textcolor{red}{Distribuer des logiciels gratuitement est une des
      principales méthodes pour infecter des ordinateurs}
  \end{center}

\end{frame}


\begin{frame}
  \frametitle{Anti-Virus}

  \vfill

   \begin{block}{Fonctionnement}
     \begin{itemize}
     \item lit un exécutable pour retrouver des schémas d'appels de
       fonction ou de boucles, tests utilisés par des virus connus
     \item base virale: liste des virus à tester, toujours incomplète:
       \begin{itemize}
       \item des virus sont enlevés lorsqu'ils ne sont plus utilisés
       \item des virus pas encore découverts ne sont pas testés
       \end{itemize}
     \end{itemize}
  \end{block}

  \vfill

  \begin{block}{Virus modernes}
    \begin{itemize}
    \item lors de la compilation, plein de variantes trompant les
      algorithmes de détection
    \item les anti-virus risquent de devenir complètement inefficaces
    \item il leur est aussi possible de passer à travers beaucoup de pare-feux
    \end{itemize}
  \end{block}

  \vfill



\end{frame}



\begin{frame}
  \frametitle{Mises à Jour}

  \vfill

   \begin{block}{Failles 0-day}
     \begin{itemize}
     \item failles qui sont exploitées avant d'être découvertes officiellement
     \item Computer Emergency Response Teams: dès qu'une telle faille
       est découverte, patch fourni le plus vite possible (avec
       explication sommaire du problème)
     \end{itemize}
  \end{block}

  \vfill
  \begin{block}{Conséquence}
    \begin{itemize}
    \item dès qu'un patch est disponible, toutes les menaces possibles
      sont au courant de la vulnérabilité
    \item si cette vulnérabilité affecte une partie du SI, il faut
      donc faire le plus rapidement possible une mise à jour pour la
      supprimer
    \end{itemize}
  \end{block}

  \vfill
\end{frame}

\begin{frame}
  \frametitle{Mises à jour automatiques}

  \vfill

   \begin{block}{Avantages}
     \begin{itemize}
     \item permet d'assurer que les mises à jour les plus récentes
       sont installées dès que possible
     \item simplifie l'administration
     \end{itemize}
  \end{block}

  \vfill

  \begin{block}{Inconvénients}
    \begin{itemize}
    \item le fonctionnement du SI est de la responsabilité de
      l'administrateur
    \item plus sûr pour le fonctionnement: MàJ sur machine test,
      vérification des fonctionnalités, sauvegarde de chaque machine
      en production avant MàJ
    \end{itemize}
  \end{block}

  \vfill



\end{frame}

\begin{frame}
  \frametitle{Mises à jour en pratique}

  \vfill

   \begin{block}{Patch Tuesday (en cours d'abandon ?)}
     \begin{itemize}
     \item politique de Microsoft: déployer les patchs de sécurité à
       travers Windows Server Update Service (maintenant Windows
       Update for Business) à une date fixe
     \item le second mardi du mois
     \item problème possible: faille révélée le lendemain
     \end{itemize}
  \end{block}

  \vfill

  \begin{block}{Limites du patch immédiat}
    \begin{itemize}
    \item test + observation + deploy: peut être très long
    \item dans la plupart des entreprises, il faut compter plus de 90
      jours avant l'application d'un patch
    \end{itemize}
  \end{block}

  \vfill


\end{frame}


\begin{frame}
  \frametitle{Protections possibles}

  \vfill

   \begin{block}{Intégrité}
     \begin{itemize}
     \item faire des sauvegardes régulières
     \item limiter les accès des utilisateurs (pour que les logiciels
       qu'ils installent n'aient pas de droit sur la configuration de
       sécurité)
     \end{itemize}
  \end{block}

  \vfill

  \begin{block}{Confidentialité}
    \begin{itemize}
    \item chiffrement des données (sur SI et dans les mails)
    \item canaux de communication hors SI pour les mots de passe et clefs
    \item utilisation de logiciels spécialisés pour le stockage dans le Cloud (délégation de la sécurité au serveur)
    \end{itemize}
  \end{block}

  \vfill



\end{frame}




  

\end{reveals}
\end{document}
 


