\Titre{Organisation du cours}

\begin{document}

\begin{reveals}
		
\maketitle


\section{Présentation du cours}

\begin{frame}
  \frametitle{Organisation}

  \vfill

  \begin{block}{Cours/TD}
    \begin{itemize}
    \item présence obligatoire: QCM à chaque séance
    \item déroulé d'une séance typique:
      \begin{itemize}
      \item QCM: 10 mn, portant sur la séance précédente 
      \item explications, théorie (présentation)
      \item applications (TD)
      \end{itemize}
    \item un CC
    \item un CT
    \end{itemize}
  \end{block}

  \vfill

  \begin{block}{TP}
    \begin{itemize}
    \item un mini-projet:
      \begin{itemize}
      \item début semaine 5 (30/01/2017)
      \item fin semaine 8 (20/02/2017)
      \end{itemize}
    \item 2 séances d'application
    \end{itemize}
  \end{block}

  \vfill
\end{frame}

\begin{frame}
  \frametitle{Contenu du module}

  \vfill
  
  \begin{block}{Introduction à la Sécurité}
    \begin{itemize}
    \item Connaître le vocabulaire de base
    \item Connaître les notions de base 
    \item Apprendre à structurer l'analyse de sécurité
    \end{itemize}
  \end{block}

  \vfill

  \begin{block}{Beaucoup de choses différentes à apprendre !}
    \begin{itemize}
    \item La culture générale (en sécurité) s'acquiert en lisant beaucoup
    \item Reformulation: le but de ce cours est d'aider à comprendre ces lectures 
    \end{itemize}
  \end{block}

\end{frame}


\begin{frame}
  \frametitle{Ressources locales}

  \vfill
  \begin{block}{Page web du cours}
    \begin{center}
      \url{http://www.irit.fr/~Yannick.Chevalier/Cours/L3-Securite/}
    \end{center}
    Mise à jour régulièrement
  \end{block}
  \vfill

  \begin{block}{Espace moodle}
    \begin{center}
      \url{http://moodle.univ-tlse3.fr/course/view.php?id=2831}
    \end{center}
    Mise à jour régulièrement
  \end{block}

  \vfill

  \begin{block}{Ressources externes}
    \begin{itemize}
    \item  \href{https://www.wired.com/category/security/}{Wired/Security}
    \item le blog de \href{https://www.schneier.com/}{Bruce Schneier}
    \item le
      \href{https://www.ssi.gouv.fr/administration/formations/cyberedu/}{cours
        de l'ANSSI}, sur lequel s'appuie ce cours (en termes de
      contenu)
    \end{itemize}
  \end{block}

\end{frame}




\end{reveals}
\end{document}
