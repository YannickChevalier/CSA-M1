\Titre{Modèles pour le Contrôle d'accès}



  
\usepackage{ifthen}


\def\txthl#1{ \ifthenelse{\lengthtest{#1 pt<0.5pt}}{\top}{\bot} }

\begin{document}

\begin{reveals}
		
\maketitle


\begin{frame}
  \frametitle{Présentation}

  \vfill

  \begin{block}{Nomenclature}
    Vocabulaire pour parler plus préciséments de systèmes de contrôle d'accès
  \end{block}

  \vfill

  \begin{block}{Différents types de contrôle d'accès}
    \begin{enumerate}
    \item Matrice de contrôle d'accès (modèle HRU)
    \item Listes de contrôle d'accès (ACL)
    \item RBAC
    \end{enumerate}
  \end{block}

  \vfill


\end{frame}

\section{Concepts}

\begin{frame}
  \frametitle{Domaine}

  \vfill

   \begin{block}{Objet}
     \begin{itemize}
     \item élément du SI contenant ou recevant de l'information
     \item exemples: enregistrement et champ (BD), blocs, pages,
       segments (mémoire), fichiers, répertoires (système de fichier),
       programmes, périphériques (vidéo, son, réseau)
     \end{itemize}
  \end{block}

  \vfill

   \begin{block}{Sujet}
     \begin{itemize}
     \item entité causant un flux d'information entre objets
     \item exemples: \emph{processus}, personne, périphérique
     \end{itemize}
  \end{block}

  \vfill

   \begin{block}{Opération}
     \begin{itemize}
     \item séquence d'instructions demandée par un sujet
     \item exemples: lecture, écriture, exécution, opérations ReST,
       création d'une table ou d'une base de données
     \end{itemize}
  \end{block}

  \vfill


\end{frame}

\begin{frame}
  \frametitle{Permission}

  \vfill

   \begin{block}{Définition}
     Une \emph{permission} est une autorisation d'effectuer une
     opération qui affecte un objet.
  \end{block}

  \vfill

  \begin{block}{Note}
    \begin{itemize}
    \item un clerc de banque n'a pas la permission de faire des
      retraits ou des ajouts (opérations)
    \item par contre, pour chaque compte de ses clients, il a la
      permission de faire ces opérations
    \end{itemize}
  \end{block}

  \vfill



\end{frame}

\begin{frame}
  \frametitle{Matrice de contrôle d'accès}

  \vfill

   \begin{block}{modèle HRU [76]}
     Matrice dans laquelle:
     \begin{itemize}
     \item chaque ligne représente un sujet
     \item chaque colonne représente un objet
     \item chaque case contient l'ensemble des opérations sur l'objet
       qui sont autorisées pour le sujet
     \end{itemize}
  \end{block}

  \vfill

  \begin{center}
    \textcolor{blue}{Exemple:}~~~\begin{tabular}[c]{|l|c|c|c|}
      \rowcolor{white}
      \hline &\(o_1\)&\(o_2\)&\(o_3\) \\\hline
        \rowcolor{white}
    alice & r,w & r & r,x \\\hline
      \rowcolor{white}
      bob & r & r &  \\\hline
    \end{tabular}
  \end{center}


  \vfill
\end{frame}


\begin{frame}
  \frametitle{Listes de contrôle d'accès (ACL)}

  \vfill

   \begin{block}{Définition}
     \begin{itemize}
     \item liste associée à un objet qui spécifie tous les sujets
       pouvant accéder à l'objet ainsi que leurs droits
     \item Une telle liste est attachée à un objet, et ses éléments sont des couples
       \((\text{sujet},\{\text{ens. de permissions}\})\)
     \end{itemize}
  \end{block}

  \vfill

  \begin{center}
    \textcolor{blue}{Exemple:}~~~\begin{tabular}[c]{|l|c|c|c|}
      \rowcolor{white}
      \hline &\only<2>{\cellcolor{red}}\(o_1\)&\only<3>{\cellcolor{red}}\(o_2\)&\only<4>{\cellcolor{red}}\(o_3\) \\\hline
        \rowcolor{white}
    alice &\only<2>{\cellcolor{red}} r,w & \only<3>{\cellcolor{red}}r & \only<4>{\cellcolor{red}}r,x \\\hline
      \rowcolor{white}
      bob &\only<2>{\cellcolor{red}} r & \only<3>{\cellcolor{red}}r & \only<4>{\cellcolor{red}} \\\hline
    \end{tabular}
  \end{center}
  \pause
  \only<2-4>{ACL de }\only<2>{\(o_1\): [(alice,\{r,w\}),(bob,\{r\})]}\pause
  \only<3>{\(o_2\): [(alice,\{r\}),(bob,\{r\})]}\pause
  \only<4>{\(o_3\): [(alice,\{r,x\}),(bob,\(\emptyset\))]}
  \vfill


\end{frame}

\begin{frame}[fragile]
  \frametitle{ACL sous linux}

  \vfill

  \begin{block}{Syntaxe}
    \begin{lstlisting}[language=bash]
      setfacl mode <-d> type:nom:perm
    \end{lstlisting}
      \begin{itemize}
      \item mode: -m (modifier) ou -x (supprimer)
      \item -d: pour les répertoires uniquement, indique les ACL des fichiers qui seront créés
      \item type: u our user, g pour groupe
      \item nom: nom de l'utilisateur ou du groupe
      \item perm: rwx
      \end{itemize}
  \end{block}
  
  \vfill

   
\end{frame}

\begin{frame}[fragile]
  \frametitle{Capacités}
  \def\firstrowcolor{\rowcolor{white}}
  \def\secondrowcolor{\rowcolor{white}}
  \only<2>{\def\firstrowcolor{\rowcolor{red}}}
  \only<3>{\def\secondrowcolor{\rowcolor{red}}}

  \vfill
  \begin{block}{Définition}
    les \emph{capacités} d'un utilisateur sont toutes les permissions de cet utilisateur
  \end{block}

  \vfill
  \begin{block}{Exemples}
    \begin{itemize}
    \item historiquement trop lourd à implémenter dans des systèmes
      d'exploitation pour des utilisations réelles
    \item Sous Linux, chaque \emph{capabilities} est un ensembles de capacités
    \item Permissions sous Android (les applications sont des sujet)
    \item Fuchsia (OS Google, en cours de développement) semble utiliser les capacités
    \end{itemize}
  \end{block}

  \vfill   

  \begin{center}
    \textcolor{blue}{Exemple:}~~~\begin{tabular}[c]{|l|c|c|c|}
      \hline\rowcolor{white}  &\(o_1\)&\(o_2\)&\(o_3\) \\
       \hline
       \firstrowcolor alice & r,w & r & r,x \\
        \hline
        \secondrowcolor bob & r & r &  \\
         \hline
    \end{tabular}
  \end{center}
\pause
  \uncover<2-3>{Capacités de }
  \only<2>{alice: [(\(o_1\),\{r,w\}),(\(o_2\),\{r\}),(\(o_3\),\{r,x\})]}\pause
  \only<3>{bob: [(\(o_1\),\{r\}),(\(o_2\),\{r\}),(\(o_3\),\(\emptyset\))]}

  \vfill
\end{frame}


\begin{frame}
  \frametitle{Domaines et types}

  \vfill

  \begin{block}{Domaine}
    regroupement de sujets (dans un groupe ou un rôle)
  \end{block}
  
  \vfill

  \begin{block}{Type}
    regroupement d'objets
  \end{block}
  
  \vfill
  \begin{block}{Utilité}
    \begin{itemize}
    \item on peut utiliser des domaines et/ou des types pour obtenir
      une politique de contrôle d'accès plus simple
    \item exemple: domaine \(\Rightarrow\) rôle dans une application Web,
      type \(\Rightarrow\) classe d'un modèle
    \end{itemize}
  \end{block}

  \vfill



\end{frame}



\section{Modalités du contrôle d'accès}

\begin{frame}
  \frametitle{Contrôle d'accès discrétionnaire}

  \vfill

   \begin{block}{Contrôle d'accès discrétionnaire (DAC---ACL, capacités)}
     Chaque objet est la propriété d'un sujet qui définit les droits
     d'accès sur cet objet
   \end{block}

  \vfill

   \begin{block}{Contrôle d'accès mandataire (MAC---Bell-LaPadula, Biba)}
     Tous les objets sont la propriété virtuelle d'un sujet hors du
     système qui délègue les permissions aux différents sujets
   \end{block}

  \vfill

   \begin{block}{Contrôle d'accès non-discrétionnaire (nouveau)}
     Le système est en principe discrétionnaire, mais des règles
     limitent les possibilités des utilisateurs:
     \begin{enumerate}
     \item en fonction de leurs actions passées (History-BAC)
     \item en fonction de règles de séparation de tâche (SoD-BAC)
     \item en fonction de rôles (Role-BAC, RBAC)
     \end{enumerate}
     En général, on adopte en fonction du contexte un mélange de ces
     différents types de politiques.
   \end{block}

  \vfill


\end{frame}

 

\begin{frame}
  \frametitle{Évaluation DAC}

  \vfill

   \begin{block}{Limites des ACL}
     \begin{itemize}
     \item il est facile de savoir qui a accès à une ressource donnée
     \item il est très difficile, voire impossible, de connaître les permissions d'un sujet
     \end{itemize}
  \end{block}

  \vfill

   \begin{block}{Limites des capacités}
     \begin{itemize}
     \item il est facile de savoir les ressources auxquelles un sujet a accès
     \item il est très difficile, voire impossible, de connaître les permissions associées à un objet
     \end{itemize}
  \end{block}

  \vfill
  \begin{block}{Suite}
    Exploration de systèmes non-mandataires plus adaptés pour la
    sécurisation de grands systèmes d'information
  \end{block}

  \vfill
\end{frame}


\section{RBAC}

\begin{frame}
  \frametitle{RBAC}

  \vfill

  \begin{block}{Types de RBAC}
    \begin{itemize}
    \item Core: rôle \(=\) groupe
    \item Hiérarchique: introduction d'une hiérarchie de rôle (et d'héritage des permissions)
    \item Contraint: introduction de contraintes de séparation de
      tâche (un sujet ne peut pas jouer dans 2 équipes pendant un même match)
    \item On ne s'intéresse qu'aux deux premiers cas
    \end{itemize}
  \end{block}

  \vfill



\end{frame}


\begin{frame}
  \frametitle{Core RBAC}

  \vfill

   \begin{block}{Principe}
     \begin{itemize}
     \item Classification des sujets: dès qu'il est authentifié, un
       sujet acquiert plusieurs rôles (relation \texttt{Role} entre
       sujets et rôles)
     \item Politique de contrôle d'accès: une relation \texttt{Perm}
       relie les rôles, les actions, et les objets
     \item Application: un sujet \(s\) a la permission de faire
       l'opération \(a\) sur l'objet \(o\) si \(s\) a un rôle \(r\)
       tel que \(\text{\tt Perm}(r,a,o)\) est vrai
     \end{itemize}
  \end{block}

  \vfill
  
  \begin{block}{Formalisation logique}
    \[
      \text{\tt Autorise}(s,a,o) \leftarrow \text{\tt Role}(s,r),\text{\tt Perm}(r,a,o)
    \]
  \end{block}

  \vfill
\end{frame}

\begin{frame}
  \frametitle{RBAC hiérarchique}
  \framesubtitle{En plus de Core}

  \vfill

   \begin{block}{Principe}
     \begin{itemize}
     \item héritage: si \(r \le r'\) alors \(r'\) peut faire tout ce que peut faire \(r\)
     \item On peut de passer de la relation \texttt{Role} en disant
       que tout sujet \(s\) a un rôle \(s\) (c'est le cas pour les bases de données)
     \item Politique de contrôle d'accès: une relation \texttt{Perm}
       relie les rôles, les actions, et les objets
     \item Application: un sujet \(s\) a la permission de faire
       l'opération \(a\) sur l'objet \(o\) si \(s\) a un rôle \(r\)
       \textbf{et qu'il existe un rôle \(r'\le r\)} tel que
       \(\text{\tt Perm}(r',a,o)\) est vrai
     \end{itemize}
  \end{block}
  
  \vfill
  
  \begin{block}{Formalisation logique}
    \[
      \text{\tt Autorise}(s,a,o) \leftarrow \text{\tt Role}(s,r),\text{\tt Perm}(r',a,o),r'\le r
    \]
  \end{block}

  \vfill
\end{frame}

\begin{frame}
  \frametitle{Limitations de RBAC}

  \vfill

   \begin{block}{Ingénierie des rôles}
     \begin{itemize}
     \item l'écriture de politiques RBAC (hiérarchiques) suppose une
       certaine uniformité
     \item i.e., le passage par la relation \texttt{Role} doit réduire le nombre de
       permissions \texttt{Perm} à gérer
     \item ingénierie des rôles: trouver les rôles qui permettent la
       meilleure réduction possible
     \end{itemize}
  \end{block}

  \vfill
  \begin{block}{Politique des moindres privilèges}
    \begin{itemize}
    \item but de la mise en place d'un système de contrôle accès: que
      chaque sujet ne puisse faire que ce dont il a besoin
    \item tentation (compréhensible) de simplifier le contrôle d'accès
      en ayant des rôles plus larges que nécessaire pour regrouper
      (uniformiser) différents cas
    \item d'où un conflit avec l'ingénierie des rôles
    \end{itemize}
  \end{block}

  \vfill

\end{frame}





\end{reveals}
\end{document}
 

