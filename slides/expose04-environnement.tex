\Titre{Menaces \& Vulnérabilités}


\setbeamercovered{invisible}
  
\usepackage{ifthen}


\def\txthl#1{ \ifthenelse{\lengthtest{#1 pt<0.5pt}}{\top}{\bot} }

\begin{document}

\begin{reveals}
		
\maketitle


\section{Introduction}

\begin{frame}
  \frametitle{Protéger de qui ?}

  \vfill

  \begin{block}{Ce qu'on a vu}
    \begin{enumerate}
    \item décomposition: on définit la sécurité par l'absence
      d'influence anormales
    \item niveaux de protection: on définit les influences normales et
      celles qu'il faut contrôler
    \item contrôle d'accès/authentification: on définit les personnes
      aux frontières du SI
    \end{enumerate}
  \end{block}

  \vfill

  \begin{block}{Aujourd'hui}
    \begin{itemize}
    \item définition des menaces, attaques, et autres
    \item tour d'horizon de ce qui existe en terme d'attaques
    \item définition des attaquants possibles
    \end{itemize}
  \end{block}

  \vfill
\end{frame}

\begin{frame}
  \frametitle{Plan}

  \vfill
  \begin{enumerate}
  \item Vulnérabilité
  \item Menace
  \item Attaque
  \end{enumerate}
  \vfill
\end{frame}


\begin{frame}
  \frametitle{Vulnérabilités}

  \vfill
  
  \begin{block}{Point de vue abstrait}
    flux d'information anormal possible et pas traité par une
    procédure de classification/déclassification (entre les
    composants) ni par le contrôle d'accès (pour les sujets)
  \end{block}

  \vfill

  \begin{block}{Point de vue concret}
    Essentiellement deux cas possibles:
    \begin{itemize}
    \pause\item Y'a un bug: l'exploitation d'un bug permet des flux
    d'information non prévus 
    \pause\item Y'a un trou dans le mur: tous les cas n'ont pas été
    recensés, ou certains recensés ont été mal traités
    \end{itemize}
  \end{block}

  \vfill
\end{frame}


\begin{frame}
  \frametitle{Menace}

  \vfill

  \begin{block}{Définition}
    toute chose ou personne dont l'effet est un dommage au système à protéger
  \end{block}

  \vfill

  \begin{block}{Exemples}
    \begin{itemize}
    \item pluie si une partie du site est en zone inondable
    \item employé mécontent
    \item concurrent
    \end{itemize}
    suite: focus sur les menaces représentées par des humains
  \end{block}
\end{frame}

\begin{frame}
  \frametitle{Attaque}

  \vfill
  \begin{block}{Définition}
  Une attaque de sécurité est l'exploitation d'une vulnérabilité
  par un acteur (personne, entreprise, État)
  \end{block}
  \vfill
  \begin{block}{Points à considérer}
    \begin{itemize}
    \item concrétisation d'une menace humaine
    \item buts (gain \textit{vs.} perte) et opportunité (facilité à
      mener une attaque) à prendre en compte dans la mesure de dangerosité
    \end{itemize}
  \end{block}


  \vfill

  \begin{center}
  \alert{Il est important d'identifier les acteurs possibles}
  \end{center}
  
  \vfill

\end{frame}

\section{Vulnérabilités}

\begin{frame}
  \frametitle{Vulnérabilités logicielles}

  \vfill

  \begin{block}{Cas non traités}
    \begin{itemize}
    \item pas d'analyse de sécurité spécifique
    \item analyse de sécurité partielle
    \end{itemize}
  \end{block}

  \pause\vfill

  \begin{block}{Erreurs dans le système de sécurité}
    \begin{itemize}
    \item erreur de conception
    \item erreur d'implémentation
    \end{itemize}
  \end{block}

  \vfill

\end{frame}




\begin{frame}
  \frametitle{Analyse de sécurité partielle}

  \vfill

  \begin{block}{Modèle}
    Système à sécuriser \(=\) graphe:
    \begin{itemize}
    \item les n\oe uds sont les composants
    \item les arcs sont les flux d'information entre composants
    \end{itemize}
  \end{block}

  \vfill

  \begin{block}{Pour la personne protégeant le SI:}
    \begin{itemize}
    \item vue du SI composant par composant
    \item focus sur les arcs menant aux composants les plus importants
    \item manque de moyens: oubli des autres
    \end{itemize}
  \end{block}

  \vfill

  \begin{block}{Pour une personne attaquant le SI}
    \begin{itemize}
    \item les attaques directes sont rares !
    \item en général: compromissions successives de
    composants du SI
    \item un attaquant recherche d'un chemin dans le graphe qui l'amène
    d'un n\oe ud contrôlé vers un n\oe ud but
    \end{itemize}
  \end{block}


\end{frame}


\begin{frame}
  \frametitle{Erreurs de conception}

  \vfill
  \begin{block}{Protocoles}
    \begin{itemize}
    \item hypothèses non raisonnables (\textit{e.g.}, cryptographie parfaite)
    \item erreurs logiques dans la spécification du protocole (\textit{cf.} TD)
    \end{itemize}
  \end{block}

  \vfill

  \begin{block}{Contrôle d'accès}
    \begin{itemize}
    \item erreurs dans l'écriture de la politique de contrôle d'accès
      \begin{center}
        \textcolor{red}{Pour certains systèmes d'entreprises, une politique pouvant être décrite en quelques lignes demande des pages d'écriture de règles}
      \end{center}
    \item erreurs dans les règles de filtrage des pare-feux
    \end{itemize}
  \end{block}

\vfill

\begin{block}{Morale}
  \begin{itemize}
  \item la conception de composants pour la sécurité demande une
    grande expertise
  \item toujours préférable de réutiliser des solutions existantes
  \end{itemize}
\end{block}

\vfill
\end{frame}

\begin{frame}
  \frametitle{Erreurs d'implémentation (1/2)}

  \vfill

  \begin{block}{Erreurs logicielles}
    \begin{itemize}
    \item heartbleed: attaque sur une implémentation de TLS
    \item buffer overflow
    \item \(\ldots\)
    \end{itemize}
  \end{block}

  \vfill

  \begin{block}{Contre-mesures}
    \begin{itemize}
    \item utilisation de flags lors de la compilation (pour gcc,
      \texttt{-fstack-protector-strong} alerte en cas de buffer
      overflow)
    \item relecture de code, documentation, ingénierie logicielle
    \end{itemize}
  \end{block}

\end{frame}

\begin{frame}
  \frametitle{Erreurs d'implémentation (2/2)}

  \vfill

  \begin{block}{Erreurs spécifiques à la sécurité}
    \begin{itemize}
    \item mise à zéro d'un tableau enlevée lors de l'optimisation
    \item différences de temps d'exécution dans différents cas qui
      permettent de remonter aux valeurs dans un programme
    \item \(\Rightarrow\) certaines failles ne sont pas capturables
      par le développement logiciel classique
    \end{itemize}
  \end{block}

  \vfill

  \begin{block}{Contre-mesures}
    Utilisation de code développé par des experts (par exemple, libsodium)
  \end{block}

\end{frame}


\section{Menaces}

\begin{frame}
  \frametitle{Importance}

  \vfill

  \begin{block}{De la vulnérabilité à l'attaque}
    opportunité d'exploitation d'une vulnérabilité dépend:
    \begin{itemize}
    \item des possibilités d'accès des acteurs possibles
    \item gain possible de cette exploitation pour chaque acteur
      identifié
    \item mitigé par le coût anticipé de cette exploitation pour
      chaque acteur identifié
    \end{itemize}
  \end{block}
  \vfill

  \begin{block}{Identification d'une menace}
    \begin{itemize}
    \item qui: quelle puissance de calcul ? quel accès de base donné
      par le système de contrôle d'accès ?
    \item pourquoi: balance des risques: gains \textit{vs} perte
    \item but: au sein du SI ou au-delà ?
    \end{itemize}
  \end{block}

\end{frame}

\begin{frame}
  \frametitle{Qui ?}

\vfill
\begin{center}
  \begin{tabular}{|c|c|c|c|}
    \hline
    & puissance & accès & risques \\\hline
    état & +++ & +++ & + \\\hline
    concurrent & ++ & + & ++ \\\hline
    employé & + & +++ & +++ \\\hline
    cyber-crime& + à ++& + à ++ & +\\\hline
  \end{tabular}
\end{center}

\vfill
\begin{block}{Remarques:}
  \begin{itemize}
  \item peu de rétorsions possible contre un état étranger, l'accès
    direct est en général faible
  \item un concurrent peu avoir (par des relations communes) accès à
    certaines données, rétorsions contre un concurrent étranger faibles
  \item un employé part avec beaucoup plus de données, mais n'a pas
    beaucoup de moyens
  \end{itemize}
\end{block}
\vfill
\end{frame}

\begin{frame}
  \frametitle{Sécuriser en augmentant les risques/la difficulté}

  \vfill

  \begin{block}{Raisonnement}
    Si la menace est motivée par un rapport bénéfice/coût, on peut 
    augmenter le coût pour réduire l'opportunité de mener l'attaque
  \end{block}

  \vfill

  \begin{block}{Exemples}
    \begin{itemize}
    \item logs, surveillance vidéo: preuve d'implication d'une personne
    \item cryptographie: obligation de mise en \oe uvre de moyens chers
    \item fouille, sas, carte d'identité: contournements chers
    \end{itemize}
  \end{block}

  \vfill

  \begin{block}{Mais...}
    \begin{itemize}
    \item la mise en place de solutions de sécurité coûte cher
    \item \textcolor{red}{audit:} valider le coût d'une solution par rapport à la perte causée par une attaque
    \item exemple: pendant longtemps, attaque connue (et utilisée)
      mais non corrigée sur les cartes bleues car la solution aurait
      coûter plus cher que la fraude
    \end{itemize}
  \end{block}
  \vfill
\end{frame}
\begin{frame}
  \frametitle{Dans quel but ?}

  \vfill

  \begin{block}{Dans le cadre du recensement des composants}
    \begin{itemize}
    \item ressources interne à protéger (cartes d'accès, données de
      l'entreprise)
    \item ressources permettant d'accéder à des ressources externes
      (vol de cartes d'identités vierges chez un imprimeur)
    \item cloud: le rôle de l'hébergeur est aussi de sécuriser les
      données de ses clients
    \end{itemize}
  \end{block}

  \vfill

  \begin{block}{Évaluation des moyens pouvant être mis en \oe uvre}
    \begin{itemize}
    \item coût \(<\) bénéfices potentiels
    \item évaluation par rapport aux accès possibles, pas par rapport
      au statut (la RDA visait principalement les secrétaires pour
      obtenir des informations, pas les haut-fonctionnaires)
    \end{itemize}
  \end{block}

\end{frame}

\begin{frame}
  \frametitle{Fraude interne}

  \vfill

  \begin{block}{Catégories de fraudeur}
    occasionnel, récurrent (de petites sommes à chaque fois), professionnel (se fait embaucher pour commettre une fraude), organisé (en groupe)
  \end{block}

  \vfill

  \begin{block}{Vulnérabilités}
    faibles procédures de contrôle interne et de surveillance des opérations, contrôle d'accès trop permissif, absence de séparation de tâche
  \end{block}

  \vfill

  \begin{block}{Fraudes recensées}
    détournement des avoirs de la clientèle ou de l'entreprise,
    création de fausses opérations, falsification des objectifs pour
    un gain de rémunération
  \end{block}

\end{frame}

\section{Attaques}

\begin{frame}

  \vfill

  \Large cf. TD
  
  \vfill
\end{frame}


\end{reveals}
\end{document}


