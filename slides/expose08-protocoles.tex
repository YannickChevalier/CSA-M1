\Titre{Protocoles Cryptographiques}



  
\usepackage{ifthen}


\def\txthl#1{ \ifthenelse{\lengthtest{#1 pt<0.5pt}}{\top}{\bot} }

\begin{document}

\begin{reveals}
		
\maketitle

\section{Introduction}

\begin{frame}
  \frametitle{Protocoles \textit{vs.} Primitives}

  \vfill

   \begin{block}{Primitives}
     \begin{itemize}
     \item opérations de transformations sur les données
     \item des garanties sont fournies sur les transformations
       possibles et impossibles
     \end{itemize}
  \end{block}

  \vfill

  \begin{block}{Protocoles cryptographiques}
    \begin{itemize}
    \item séquence de messages dont le contenu a été produit par
      l'application de primitives cryptographiqes
    \item un protocole cryptographique a un \textbf{but}, qui est une
      assurance donnée à un participant s'il respecte les règles du protocole
    \end{itemize}
  \end{block}

  \vfill

  \begin{block}{Suite du cours}
    \begin{itemize}
    \item buts habituels
    \item notation simple pour les protocoles cryptographiques
    \item analyse basique d'un protocole cryptographique
    \end{itemize}
  \end{block}

  \vfill


\end{frame}

\section{Buts}

\begin{frame}
  \frametitle{Analyse d'une communication}

  \vfill
  \begin{center}
  \begin{tikzpicture}[%
    role/.style = { rounded corners= 1ex , fill=blue!30,draw,inner sep=1ex},
    channel/.style = { fill=red!30,draw, inner sep=1ex},
    mycloud/.style = { inner sep=1.5em,
      cloud,cloud puffs=19,cloud ignores aspect,
      minimum width=5cm,minimum height=3.5cm,
      align=center, 
      draw,fill=white,
    }
    ]
    \node[mycloud] (cloud) at (4,0) {} ;
    \node (internet) at (4,0.8) {Internet \ensuremath{\bot/\bot}} ;
    \node[role] (alice) at (0,0) {Alice \ensuremath{\top/\top}} ;
    \node[role] (bob) at (8,0) {Bob \ensuremath{\top/\top}} ;
    \node[channel] (ch) at (4,0) {canal de communication \ensuremath{\top/\top}};
    \draw[<->] (alice.east) -- (ch.west) ;
    \draw[<->] (ch.east) -- (bob.west) ;
  \end{tikzpicture}
  \end{center}

  \vfill

  \begin{block}{Communication à travers Internet}
    \begin{itemize}
    \item par défaut, aucune sécurité sur Internet
    \item protocole cryptographique \(=\) mise en place d'un canal
      protégeant les communications du reste d'Internet (séparation)
    \item Buts possibles:
      \begin{itemize}
      \item intégrité: pas d'écriture sur le canal à partir d'Internet
      \item confidentialité: pas de lecture sur le canal à partir d'Internet
      \end{itemize}
    \end{itemize}
  \end{block}

  \vfill



\end{frame}

\begin{frame}
  \frametitle{Formulation locale des buts}

  \vfill

   \begin{block}{Expression des buts}
     \begin{itemize}
     \item selon les protocoles, le pair n'est pas toujours connu
     \item un sujet ne doit poser des buts que sur ses actions propres
     \end{itemize}
  \end{block}

  \vfill

  \begin{block}{Pseudonymes et identité}
    \begin{itemize}
    \item identité: identifiant global d'un sujet
    \item pseudonyme: identifiant local (à une exécution du protocole) d'un sujet:
      \begin{itemize}
      \item numéro de session d'un client
      \item hôte et ports de communication
      \item pseudonyme obtenu par un protocole dédié
      \end{itemize}
    \item identité cas particulier de pseudonyme (utilisation d'une
      identité globale dans une session)
    \end{itemize}
  \end{block}

  \vfill

\end{frame}


\begin{frame}
  \frametitle{Buts pour les protocoles cryptographiques}

  \vfill

  \begin{block}{Intégrité}
    \begin{itemize}
    \item d'après les niveaux d'intégrité, seulement lors de la
      réception de messages par un sujet
    \item contenu \textit{a priori} inconnu: la propriété se réduit à
      demander que le message a été envoyé par sujet désigné
    \item identité inconnue: utilisation possible de pseudonymes
    \end{itemize}
  \end{block}

  \vfill

  \begin{block}{Confidentialité}
    \begin{itemize}
    \item d'après les niveaux de confidentialité, seulement lors de 
      l'émission de messages par un sujet
    \item dans ce cas, la propriété se réduit à demander que le
      message ne puisse être connu que par des sujets désignés
    \item identité inconnue: utilisation possible de pseudonymes
    \end{itemize}
  \end{block}

  \vfill

   
\end{frame}



\begin{frame}
  \frametitle{Authentification}

  \vfill

  \begin{block}{Preuve d'authentification}
    \begin{itemize}
    \item environnement distribué
    \item authentification repose sur une preuve
    \item la preuve ne peut être que le message reçu, ou une partie de
      ce message
    \end{itemize}
  \end{block}

  \vfill
  \begin{block}{Challenge/réponse}
    \begin{itemize}
    \item déjà vu pour les protocoles de transmission de mot de passe
    \item un sujet crée une valeur aléatoire \(r\)
    \item la présence de cette valeur dans un message doit garantir
      son origine
    \end{itemize}
  \end{block}

  \vfill
   
\end{frame}

\begin{frame}
  \frametitle{Rejeu (replay)}

  \vfill

   \begin{block}{But d'authentification naïf}
     \(A\) authentifie \(B\) en se basant sur la preuve \(Na\) si,
     quand \(A\) reçoit un message contenant \(Na\), ce message a
     précédement été envoyé par \(B\)
   \end{block}

  \vfill

  \begin{block}{Rejeu}
    rejouer un message signifie, pour un attaquant:
    \begin{itemize}
    \item enregistrer le déroulement du protocole
    \item réutiliser ces messages pour se faire passer pour un des
      participants
    \end{itemize}
  \end{block}

  \vfill

  \begin{block}{Formulation naïve et rejeu}
    \begin{itemize}
    \item lors d'un rejeu, le message reçu a été précédement envoyé
      par l'auteur légitime
    \item pour l'authentification, il faut compter le nombre de fois
      que le message a été envoyé/reçu
    \end{itemize}
  \end{block}

  \vfill

\end{frame}

\begin{frame}
  \frametitle{Buts d'authentification}

  \vfill

   \begin{block}{Authentification faible}
     \(A\) authentifie \(B\) en se basant sur la preuve \(Na\) si,
     quand \(A\) reçoit un message contenant \(Na\), ce message a
     précédement été envoyé par \(B\)    
  \end{block}

  \vfill

   \begin{block}{Authentification forte}
     \(A\) authentifie \(B\) en se basant sur la preuve \(Na\) si, le
     message contenant \(Na\) a été reçu par \(A\) moins souvent qu'il
     n'a été envoyé par \(B\)
  \end{block}

  \vfill


\end{frame}

\begin{frame}
  \frametitle{Confidentialité}

  \vfill

   \begin{block}{Besoin d'authentification}
     \begin{itemize}
     \item l'information envoyée est destinée à certaines personnes
     \item il faut donc avoir la certitude de l'identité (ou de son
       pseudonyme) d'un pair avant de lui envoyé une donnée
       confidentielle
     \end{itemize}
  \end{block}

  \vfill

  \begin{block}{But de confidentialité}
    Une partie d'un message envoyé par \(A\) ne peut être lue que par
    \(B_1,\ldots,B_n\)
  \end{block}

  \vfill
\end{frame}


\begin{frame}
  \frametitle{Spécification simplifiée d'un protocole}

  \vfill

   \begin{block}{Parties utilisées}
     \begin{description}
     \item[connaissances initiales:] pour chaque sujet, une liste de valeurs connues
     \item[échange:] une suite de communications
       \[
         \begin{array}{rc@{~\ensuremath{\rightarrow}~}c@{~:~}l}
         1. & A & B& M_1 \\
         2. & B & A& M_2 \\
           \multicolumn{4}{c}{\ensuremath{\vdots}}\\
         \end{array}
       \]
     \item[Buts:] la description des buts de confidentialité et d'authentification
     \end{description}
  \end{block}

  \vfill
  \begin{block}{Note}
    Pour simplifier, on utiliser \(\lbrace\_\rbrace\_\) pour le
    chiffrement symétrique, asymétrique, et la signature digitale. La
    clef utilisée indique l'opération utilisée.
  \end{block}

  \vfill

\end{frame}






\section{TLS}

\begin{frame}
  \frametitle{Historique}

  \vfill

   \begin{block}{SSL}
     \begin{itemize}
     \item débuts d'internet
     \item protocole proposé par Netscape (adresses \texttt{https})
     \item plusieurs versions (v1,v2,v3)
     \item toutes buggées
     \end{itemize}
   \end{block}

  \vfill
  \begin{block}{TLS}
    \begin{itemize}
    \item standardisation IETF de SSL
    \item TLS 1.0 = SSL v3 (sauf détails mineurs)
    \item version courante 1.2 (théorie), 1.1 en pratique
    \end{itemize}
  \end{block}

  \vfill
\end{frame}

\begin{frame}
  \frametitle{Fonctionnement de TLS}

  \vfill

   \begin{block}{Négociation}
     \begin{itemize}
     \item le client et le serveur s'entendent sur la version à utiliser
     et  sur les algorithmes à utiliser dans les phases suivantes
     \item risque: attaque demandant d'utiliser une version buggée
     \end{itemize}
  \end{block}

  \vfill
  \begin{block}{Rendez-vous (handshake)}
    \begin{itemize}
    \item phase d'authentification
    \item négociation d'un contexte de sécurité (clef secrète)
    \end{itemize}
  \end{block}

  \vfill
  \begin{block}{Utilisation}
    \begin{itemize}
    \item chiffrement symétrique des messages échangés basé sur le contexte de sécurité
    \item on chiffre le flux de message, pas les messages individuels 
    \end{itemize}
  \end{block}

  \vfill

  \begin{block}{Fin/renégociation}
    \begin{itemize}
    \item lorsque la période de validité du contexte de sécurité se termine
    \item ou à la demande d'un des 2 participants
    \end{itemize}
  \end{block}

  \vfill
\end{frame}


\begin{frame}[fragile]
  \frametitle{Négociation}

  \vfill

   \begin{block}{Négociation initiale}
     \begin{itemize}
     \item dans le protocole HTTP, demande de changement du protocole
       de transport (client ou serveur, mot-clef \texttt{Connection})
     \item valeurs du champ \texttt{upgrade}: algorithmes supportés par le
       navigateur, par ordre de préférence
     \end{itemize}
  \end{block}

  \vfill

\begin{verbatim}
GET /hello.txt HTTP/1.1
Host: www.example.com
Connection: upgrade
Upgrade: HTTP/2.0, SHTTP/1.3, IRC/6.9, RTA/x11
\end{verbatim}
  
\end{frame}

\begin{frame}
  \frametitle{Rendez-vous (Handshake)}

  \vfill

   \begin{block}{Principe}
     \begin{itemize}
     \item protocole d'authentification mutuelle, utilisation de
       certificats (voir plus loin)
     \item Diffie-Hellman le plus utilisé
     \item \textbf{mais} beaucoup d'implémentations (\(\sim\) 90\%)
       avec de mauvais paramètres
     \end{itemize}
  \end{block}

  \vfill
  \begin{block}{Protocole de Diffie-Hellman}
    \(A\) connait \(A,B,g,N,Na\), et \(B\) connait \(A,B,g,N,Nb\)
    (\(Na\), \(Nb\) aléatoires)
    \[
      \begin{array}{cr@{\ensuremath{\rightarrow}}c@{:}l}
        1.& A & B & g^{Na} \mod N \\
        2. & B & A & g^{Nb} \mod N\\
      \end{array}
    \]
    But: \(g^{Na\times Nb} \mod N\) est connu seulement de \(A\) et \(B\)
  \end{block}

  \vfill

\end{frame}

\begin{frame}
  \frametitle{Explications Diffie-Hellman}

  \vfill

   \begin{block}{Calcul du secret}
     \begin{itemize}
     \item \(g^{Na\times Nb} \mod N = {g^{Na}}^{Nb} \mod N\)
     \item \(g^{Na\times Nb} \mod N = {g^{Nb}}^{Na} \mod N\)
     \end{itemize}
  \end{block}

  \vfill

  \begin{block}{Robustesse (jeu)}
    \begin{itemize}
    \item on doit au joueur soit:
      \begin{itemize}
      \item \((g^{Na} \mod N,g^{Nb} \mod N,g^{Na\times Nb} \mod N)\)
      \item \((g^{Na} \mod N,g^{Nb} \mod N,g^{r} \mod N)\), où \(r\)
        est un nombre aléatoire
      \end{itemize}
    \item le joueur gagne si il devine correctement quel choix a été
      fait
    \item Hypothèse de Diffie-Hellman Décisionnel (DDH): aucun joueur
      (machine de Turing) réaliste ne peut faire mieux que pile ou
      face
    \item Si un observateur peut calculer une partie du secret à
      partir de l'échange, il peut faire mieux que une chance sur deux
    \end{itemize}
  \end{block}

  \vfill

\end{frame}

\begin{frame}
  \frametitle{Utilisation}

  \vfill

  \begin{block}{Chiffrements de flux}
    \begin{itemize}
    \item soit chiffrement type Vernam, avec clair \(\oplus\) nombre
      aléatoire
    \item soit chiffrement par bloc: la même clef est toujours
      utilisée, mais le block précédent est pris en compte
    \item il y a une attaque générique sur les chiffrements par blocs
      (mais faible probabilité de succès)
    \end{itemize}
  \end{block}

  \vfill
  \begin{block}{Algorithmes utilisés}
    \begin{itemize}
    \item par block: RCA, 3DES (seuls disponibles sur XP),
      AES,\(\ldots\)
    \item par flux: Salsa, Chacha, blowfish,\(\ldots\)
    \end{itemize}
  \end{block}

  \vfill

\end{frame}


\section{PKI}

\begin{frame}
  \frametitle{Retour sur Diffie-Hellman}

  \vfill

   \begin{block}{Pas d'authentification !}
     \begin{itemize}
     \item un attaquant \(C\) peut s'immiscer dans un échange entre
       \(A-B\) pour le remplacer par deux échanges \(A-C\) et \(C-B\)
     \item on parle d'attaque par un intermédiaire
       (\emph{man-in-the-middle})
     \item il y a bien un secret partagé, mais on ne sait pas avec qui
     \end{itemize}
  \end{block}

  \vfill

  \begin{block}{Ajout de l'authentification}
    \begin{itemize}
    \item pour authentifier la session, les messages sont signés par
      le client et le serveur en:
      \begin{itemize}
      \item utilisant une clef de signature
      \item la clef de validation correspondante est envoyée en même
        temps
      \end{itemize}
    \item problème: comment relier la clef à une identité ?
    \end{itemize}
  \end{block}

  \vfill


\end{frame}

\begin{frame}
  \frametitle{Certificats}

  \vfill
  \begin{block}{Certificat}
    Un certificat est un document signé digitalement par un sujet.
  \end{block}

  \vfill



   \begin{block}{Utilisation}
     \begin{itemize}
     \item un sujet s'engage sur la véracité d'informations en signant
       ces informations
     \item 2 cas possibles:
       \begin{itemize}
       \item soit le lecteur connait la clef de validation permettant
         de valider la signature
       \item soit le lecteur doit obtenir la preuve que la clef à
         utiliser pour la validation est bien celle du sujet
       \end{itemize}
     \end{itemize}
  \end{block}

  \vfill


\end{frame}

\begin{frame}
  \frametitle{Infrastructures de clefs publiques}

  \vfill

  \begin{block}{But: propagation de la confiance}
    \begin{itemize}
    \item confiance: ensemble de certificats justifié par des certificats connus du lecteur
    \item http: justification existe \(=\) cadenas vert, pas de justification
      \(=\) cadenas rouge/avertissement
    \end{itemize}
  \end{block}

  \vfill

  \begin{block}{Confiance dans les navigateurs}
    \begin{itemize}
    \item chaque navigateur a une liste des certificats reconnus
      (certificats \emph{racines} pour la confiances)
    \item pour être accepté, un site Internet doit obtenir un
      certificat qui peut être validé pour un certificat racine
    \end{itemize}
  \end{block}

  \vfill
\end{frame}

\begin{frame}
  \frametitle{Utilité et fragilité}

  \vfill

   \begin{block}{Problème d'initialisation}
     il faut avoir confiance dans les entreprises qui émettent des
     certificats racines
   \end{block}

  \vfill
  \begin{block}{Exemple: navigateurs d'entreprise}
    \begin{itemize}
    \item des sociétés sont spécialisées dans l'émission de ``faux'' certificats
    \item ces certificats sont mis par des entreprises dans le
      navigateur de leurs employés
    \item celà permet à la société émettrice d'intercepter les
      communications https des employés (attaque Man-in-the-Middle par
      la société)
    \item avantage/désavantage: une entreprise a accès à l'historique
      de navigation de ses salariés 
    \item Certificate Transparency: effort pour permettre aux
      internautes de savoir si les certificats qu'ils acceptent ont
      été réellement émis par le site
    \end{itemize}
  \end{block}

  \vfill
\end{frame}



\end{reveals}
\end{document}
 
