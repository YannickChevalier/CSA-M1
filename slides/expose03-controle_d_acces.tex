\Titre{Contrôle d'accès}



  
\usepackage{ifthen}


\def\txthl#1{ \ifthenelse{\lengthtest{#1 pt<0.5pt}}{\top}{\bot} }

\begin{document}

\begin{reveals}
		
\maketitle


\section{Principes}

\begin{frame}
  \frametitle{Système d'information décomposé}


  % \begin{center}
  %   \begin{tikzpicture}
  %     \foreach \j in {0,1,...,4}{% Two indices running over each
  %       \pgfmathsetmacro{\x}{rnd}
  %       \pgfmathsetmacro{\y}{rnd}
        
  %       \pgfmathsetmacro{\alconf}{rnd}
  %       \pgfmathsetmacro{\alint}{rnd}
  %       \node[draw,circle,inner sep=2pt,fill,label={\ensuremath{\scriptscriptstyle \txthl{\alconf}/\txthl{\alint}}}] (\j)at (\x*5,\y*5) {} ;
  %     } 
  %     \draw[->] (1) -- (2) ;
  %     \draw[->] (1) -- (4) ;
  %     \draw[->] (2) -- (0) ;
  %     \draw[->] (3) -- (2) ;
  %     \draw[->] (4) -- (3) ;
  %     \draw[->] (4) -- (1) ;
  %     \draw[->] (0) -- (4) ;
  %   \end{tikzpicture}
  %   \parbox{0.7\textwidth}{%
  %     \begin{trivlist}
  %     \item \tikz{\node[draw,circle,inner sep=2pt,fill] {}} composant
  %     \item \(\top/\bot\) niveaux pour la confidentialité et l'intégrité
  %     \item \(\rightarrow\) flux d'information
  %     \end{trivlist}
  %   }
  % \end{center}

  \vfill

  \begin{block}{But}
    \begin{itemize}
    \item recenser les \textbf{accès} possibles des sujets sur les
      objets (lire, écrire, \ldots)
    \item définir les possibilités d'\textbf{accès} des sujets aux
      composants
    \end{itemize}
  \end{block}

  \vfill

  \begin{block}{Contrôle d'accès}
    \begin{itemize}
    \item authentifier les sujets pour s'assurer de leur niveau d'habilitation/d'assurance
    \item dire quel(s) niveaux d'habilitation/assurance permettent quels accès à quels objets
    \item assurer que les accès réels aux données sont conformes 
    \end{itemize}
  \end{block}

  \vfill

\end{frame}


\section{Identification et Authentification}
 
\begin{frame}
  \frametitle{Identification}

  \vfill

  \begin{block}{Résumé}
    ensemble des attributs définissant un sujet
  \end{block}

  \vfill

  \begin{block}{Plusieurs types d'attributs possibles:}
    \begin{itemize}
    \item attributs physique: empreintes digitales, de l'\oe il, de
      l'oreille, visage, etc. \(\rightarrow\) \textbf{biométrie}
    \item contexte d'exécution
    \item possession d'objets ou de connaissances
    \end{itemize}
  \end{block}

\end{frame}

\begin{frame}
  \frametitle{Identification biométrique}

  \vfill

  \begin{block}{Avantages}
    \begin{itemize}
    \item liée à une personne physique 
    \item pas de ``délégation'' de l'identité à un objet (clef, carte d'accès)
    \item ne nécessite pas de connaissances spéciales
    \item ne nécessite pas le consentement du sujet
    \end{itemize}
  \end{block}

  \vfill

  \begin{block}{Inconvénients}
    \begin{itemize}
    \item certaines caractéristiques peuvent être copiées (soit pour
      tout le monde, soit pour une personne)
    \item lorsque c'est le cas, la personne ou le système doivent être
      désactivés de manière permanente
     \item ne nécessite pas le consentement du sujet
   \end{itemize}
  \end{block}

  \vfill

  \begin{block}{Exemple limite de biométrie}
    \begin{itemize}
    \item remplacement des captchas par des mesures du mouvement de la
      souris pour identifier si l'utilisateur d'un site web est humain
    \item les paramètres sont gardés secrets pour qu'ils ne puissent
      pas être simulés par un programme
    \end{itemize}
  \end{block}

  \vfill
\end{frame}


\begin{frame}
  \frametitle{Identification par le contexte}

  \vfill

  \begin{block}{En dehors des SI:}
    \begin{itemize}
    \item un homme a assisté gratuitement à plein d'événements en
      mettant un dossard jaune fluo
    \item \emph{The Art of Deception} (K.~Mitnick): ingénierie
      sociale, source majeure d'attaque sur les SI
    \item morale: il faut l'éviter, et pour cela, il faut éduquer les humains utilisant le SI
    \end{itemize}
  \end{block}

  \vfill

  \begin{block}{Dans les SI:}
    \begin{itemize}
    \item \texttt{/dev/securetty}: possible d'interdire le login de l'administrateur sur certaines consoles
    \item sous linux, possible d'ajouter certains groupes (voir plus
      loin) à des utilisateurs loggués à partir de certaines consoles
    \end{itemize}
  \end{block}

  \vfill

\end{frame}

\begin{frame}
  \frametitle{Identification par possession}

  \vfill

  \begin{block}{Principe:}
    \begin{itemize}
    \item on suppose que seul un sujet donné possède un certain objet
      ou une certaine connaissance
    \item cet objet ou cette connaissance permet d'identifier le sujet
    \end{itemize}
  \end{block}

  \vfill

  \begin{block}{Possession d'objet}
    \begin{itemize}
    \item carte bleue, numéro \textbf{au dos}
    \item clef
    \item carte RSA (carte qui génère un code unique valable un court
      moment)
    \end{itemize}
  \end{block}

  \vfill

  \begin{block}{Possession de connaissance}
    \begin{itemize}
    \item \textbf{mot de passe}
    \item certificat cryptographique
    \item captcha
    \end{itemize}
  \end{block}

\end{frame}

\section{Authentification}

\begin{frame}
  \frametitle{Authentification}

  \vfill

  \begin{block}{Authentification par un composant}
    \begin{itemize}
    \item un mot de passe \emph{identifie} un sujet
    \item quand un sujet se présente, on veut qu'il \emph{prouve} son identité
    \item l'authentification par un composant consiste pour ce
      logiciel à obtenir une preuve de l'identité du sujet
    \item lorsqu'il y a un risque d'écoute, la preuve consiste en
      général en un protocole challenge/réponse
    \end{itemize}
  \end{block}

  \vfill

  \begin{block}{Challenge/réponse}
    \begin{itemize}
    \item au cas où quelqu'un écouterai, ou veut une réponse différente
      à chaque fois que quelqu'un essaye de s'authentifier (re-jeu)
    \item challenge: nombre aléatoire
    \item réponse: construite à partir du mdp et du nombre aléatoire
    \end{itemize}
  \end{block}

  \vfill
  
\end{frame}

\begin{frame}
  \frametitle{Variations sur l'authentification}

  \vfill

  \begin{block}{Authentification fédérée}
    \begin{itemize}
    \item plusieurs SI indépendants se font confiance pour identifier
      leurs utilisateurs (Single Sign-On, OAuth)
    \item surtout sur le Web,  login \textit{via} Google ou Facebook, eduroam
    \item en dehors du Web, pour les réseaux de téléphonie (partage
      Free/Orange, à l'étranger)
    \end{itemize}
  \end{block}
  
  \vfill

  \begin{block}{Same Sign-on}
    plusieurs couples login/mot de passe permettent de se connecter au
    même compte (\textit{e.g.}, Office 365)
  \end{block}

  \vfill
  \begin{block}{Impact sur la sécurité}
    \begin{itemize}
    \item Single Sign-On: les organisations ne sont plus les uniques
      responsables de leurs sujets
    \item Same Sign-On: identité partagée, donc ressources partagées
      dans le cloud
    \end{itemize}
  \end{block}

  \vfill
\end{frame}

\begin{frame}
  \frametitle{Au-delà de l'authentification}

  \vfill

  \begin{block}{Non-répudiation}
    \begin{itemize}
    \item pour l'authentification, le composant se fait confiance
    \item la preuve ne peut en général pas être présentée à un tiers
      qui n'a pas confiance en le composant
    \item lorsqu'il faut pouvoir garder la preuve d'un accès (pour des
      raisons légales) indépendante du composant, on parle de
      non-répudiation
    \end{itemize}
  \end{block}

  \vfill

\end{frame}

\section{Classification des sujets}

\begin{frame}
  \frametitle{Groupes}

  \vfill

  \begin{block}{Définition}
    un \emph{groupe} est un ensemble de sujets.
  \end{block}

  \vfill

  \begin{block}{Exemples (Linux)}
  \begin{itemize}
  \item groupes administratifs: root, sys, adm, etc.
  \item groupes liés à du matériel: video, audio cdrom, etc.
  \item groupes liés à une application:www-data, irc, mail, etc.
  \end{itemize}
  \end{block}

  \vfill
\end{frame}

\begin{frame}
  \frametitle{Rôles}

  \vfill

  \begin{block}{Définition}
    Un rôle est un sujet ou un ensemble de rôles
  \end{block}

  \vfill

  \begin{block}{Cas d'utilisations}
    \begin{itemize}
    \item bases de données SQL
    \item applications Web
    \end{itemize}
  \end{block}

  \vfill

  \begin{block}{Intérêt}
    \begin{itemize}
    \item permet de hiérarchiser les permissions
    \item adapter pour les systèmes complexes
    \end{itemize}
  \end{block}

  \vfill
\end{frame}

\begin{frame}
  \frametitle{Attributs}

  \vfill

  \begin{block}{Définition}
    Caractéristique affectées à un sujet
  \end{block}

  \vfill

  \begin{block}{Intérêt}
    \begin{itemize}
    \item permet d'être plus agile qu'avec des rôles
    \item avec des attributs correspond à des permissions précises,
      il est plus facile de modifier ces permissions
    \item les capacités sous linux sont aussi utilisées pour réduire
      les permissions à celles réellement utiles (voir plus bas)
    \end{itemize}
  \end{block}

  \vfill

\end{frame}

\section{Conformité des accès}

\begin{frame}
  \frametitle{Permissions sous Unix}

  \vfill

  \begin{block}{Codage}
    \begin{itemize}
    \item entier sur 4 chiffres en base 8
    \item convention C/Unix: les entiers en base 8 commencent par un 0
    \item donc une permission est un entier \(< 07777\)
    \item les 0 en tête peuvent être enlevés
    \item les autres chiffres se lisent sous la forme \(1 +2 +4\)
    \item ex: 5 contient 1 et 4, 3 contient 1 et 2, 0 ne contient
      rien, etc.
    \end{itemize}
  \end{block}

  \vfill
\end{frame}


\begin{frame}
  \frametitle{3 derniers chiffres}

  \vfill

  \begin{block}{Lecture des permissions}
    \begin{itemize}
    \item possesseur-groupe-autre
    \item lors d'un accès, on regarde les chiffre le plus précis,
      \textbf{pas le plus avantageux}
    \end{itemize}
  \end{block}

  \vfill
  \begin{center}
    \begin{tabular}{|l|p{.4\textwidth}|p{.4\textwidth}|}
      \hline
      & fichier & répertoire \\\hline
      4 (r) & lecture & lecture du contenu\\\hline
      2 (w) & écriture & ajout, suppression, ou renommage
      des fichiers contenus\\\hline
      1 (x) & exécution comme programme & cd et recherche
      dans le répertoire\\\hline
    \end{tabular}
  \end{center}
  \vfill
\end{frame}

\begin{frame}
  \frametitle{Premier chiffre}
  \framesubtitle{ne peut être enlevé qu'avec -}

  \vfill
  \begin{block}{1/\(\pm\)t}
    \begin{itemize}
    \item le droit d'écriture sur un répertoire permet d'ajouter ou
      d'enlever des fichiers dans ce répertoire
    \item \texttt{sticky bit}: un utilisateur ne peut enlever que les
      fichiers qu'il possède
    \end{itemize}
  \end{block}
  \vfill
  \begin{block}{2/g\(\pm\)s}
    \begin{itemize}
    \item quand un programme est lancé, il effectue des accès en ayant
      le même groupe que l'utilisateur ayant lancé le programme
    \item \texttt{setgid}: si le fichier est exécutable, lors du
      lancement, le programme fera des accès en ayant le groupe du
      fichier (pas du lanceur)
    \end{itemize}
  \end{block}
  \vfill
  \begin{block}{4/u\(\pm\)s}
    \begin{itemize}
    \item quand un programme est lancé, il effectue des accès en étant
      le même sujet que l'utilisateur ayant lancé le programme
    \item \texttt{setuid} si le fichier est exécutable, lors du
      lancement, le programme fera des accès en étant le possesseur du
      fichier (pas le lanceur)
    \end{itemize}
  \end{block}

  \vfill
\end{frame}

\section{Conclusion}

\begin{frame}
  \frametitle{Au-delà des permissions classiques}

  \vfill

  \begin{itemize}
  \item utilisation des ACL: plusieurs utilisateurs et groupes par
    fichier
  \item capacités: donner certaines permissions de root à des
    programmes sans set(g,u)id root
  \item attributs de fichiers (trusted): plus de précisions sur les actions
    possibles sur un fichier
  \end{itemize}

  \vfill
\end{frame}

\end{reveals}
\end{document}
 
