\Titre{Separation}

\begin{document}

\begin{reveals}
		
\maketitle





\begin{frame}
  \frametitle{Historical background}

  \begin{block}{Before the  1970s}
    \begin{itemize}
    \item Computers only in physically secured environments
    \item Computers rebooted between each computation
    \end{itemize}
  \end{block}

  \pause
  
  \begin{block}{Origin of security concerns}
    \begin{itemize}
    \item Time-sharing: organisations with different needs use a same
      computer for economical purposes (Multics, Unix)
    \item Security goal: Time-sharing should be as secure as non-Time-sharing systems
    \end{itemize}
  \end{block}

  \pause

  \begin{block}{Examples}
    \begin{itemize}
    \item University and DoE/DoD use the same super-computer for simulation
    \item Nowadays: exactly the same concern:
      \begin{itemize}
      \item Avionics: RTCA DO-255
      \item Cloud computing
      \end{itemize}
    \end{itemize}
  \end{block}

\end{frame}

\begin{frame}
  \frametitle{Separation}

  \begin{block}{Separation informal statement}
    The execution of one program cannot influence in any way the
    execution of another program
  \end{block}

  \pause

  \begin{block}{Extensive definition}
    \begin{itemize}[<+->]
    \item A program in an infinite loop cannot preempt the processor
    \item A program with a memory leak cannot hold all the available memory
    \item In TS OS, the processor state has to be cleaned at each
      process change
    \end{itemize}
  \end{block}
\end{frame}

\begin{frame}
  \frametitle{Trusted Computing Base (1/2)}

  \begin{block}{Assume/Guarantee}
    When adding a new component to a system
    \begin{itemize}
    \item Except at the lowest level, use of other components based on assumptions  
      \begin{itemize}
      \item OS on the hardware
      \item Applications/libraries on the OS and other applications/libraries
      \item $\ldots$
      \end{itemize}
    \item The new component guarantees (at some level of certainty) some properties:
      \begin{itemize}
      \item no memory leaks,
      \item bounds on execution time,
      \item \ldots
      \end{itemize}
    \end{itemize}
  \end{block}
\end{frame}

\begin{frame}
  \frametitle{Trusted Computing Base (2/2)}

  \begin{block}{Security in CS}
    \begin{itemize}
    \item Formalisms to state and prove the guarantees provided by a
      component based on some assumptions on other components
    \item Trusted computing base: parts of a system that are not
      analysed and just assumed to provide the stated guarantees
    \item Example: processor state reset:
      \begin{itemize}
      \item register reset not in early version of Unix (unused registers were left as is)
      \item cache never resetted after a context change (leading to
        lot of security attacks, \textit{e.g.} Spectre)
      \end{itemize}
    \item problem compounded in current processors by pipeline,
      speculative execution, and on-the-fly code optimisation
    \end{itemize}
  \end{block}
\end{frame}


\begin{frame}
  \frametitle{Security Solution: Static Configuration}
  
  \begin{block}{Principle}
    Ressources are statically allocated to consumers
  \end{block}

  \begin{block} {Security implications: Access Control}
    \begin{itemize}[<+->]
    \item Access control policy: definition of the possible allocation
      of resources to consumers
    \item Must be defined before the start of the machine
    \item Cannot be changed while it is running
      \begin{center}
        \emph{Static AC}
      \end{center}
    \item Example: bounds on execution time, memory available, etc.
    \item Unmutable policy:
      \begin{center}
        \emph{Mandatory Access Control}
      \end{center}
    \item Allows for a thorough analysis of a system
    \end{itemize}
  \end{block}

\end{frame}


\begin{frame}
  \frametitle{Time separation}

  \begin{block}{Statement}
    The AC policy must specify the computation time allocated to each
    program
  \end{block}

  \begin{itemize}
  \item Hard real-time systems
  \item Allocates a fraction of the processor cycles to each process
  \item In practice:
    \begin{itemize}
    \item implemented in embedded and mission-critical systems for
      safety
      \begin{center}
        the ABS functionality cannot be affected when you start a new
        Ariana Grande song
      \end{center}
    \end{itemize}
  \end{itemize}

\end{frame}


\begin{frame}
  \frametitle{Spatial Separation}

  \begin{block}{Statement}
    Memory available to each application has to be allocated
    statically and be disjoint
  \end{block}

  \begin{itemize}
  \item \textit{a.k.a.} \emph{partitioning}
  \item Implies no direct communication between applications (IPC
    system V), communication through the OS still possible
  \item In practice
    \begin{itemize}
    \item RAM,HDD, SDD are sets of pages, and that set is partitioned,
      and each application receives one subset
    \item No malloc/mmap, or a secure variant
    \item Bounded stack
    \item No pointers unless the Processor/OS guarantees that pointers
      never access a wrong page
    \end{itemize}
  \item Very hard in practice
    \begin{itemize}
    \item Need for a specific language (\textit{e.g.} Lustre) or compiler
    \item Registers and cache still have to shared on current processors
    \end{itemize}
  \end{itemize}

\end{frame}

\begin{frame}
  \frametitle{Access Control and Spatial Separation}

  \begin{itemize}[<+->]
  \item TL;DR: very hard, but with some tolerance, achieved by current
    systems like Linux CGroups
  \item Incidentally, development driven by Cloud-hosting companies
  \item Known tolerances:
    \begin{itemize}
    \item shared libraries
    \item bounds on max memory available, but no hard partition
    \item \texttt{root} still can do whatever he wants
    \end{itemize}
  \end{itemize}

\end{frame}

\begin{frame}
  \frametitle{Partitioning in Information Systems}

  \vfill

  \begin{block}{In a computer network}
    \begin{itemize}
    \item Tasks to be performed by a system are allocated to different
      computers based on their sensitivity
    \item The network is partitioned into \emph{zones} for computers
      of similar security concerns
    \item Access Control on inter-zone exchanges (\textit{a.k.a.}
      \emph{firewall}) 
    \item Example:
      \begin{itemize}
      \item \textcolor{blue}{DMZ} at the border between a
        \textcolor{blue}{corporate network} and the
        \textcolor{blue}{outside} (3 zones), 
      \item \textcolor{blue}{Firewalls} to control packets between these zones
      \end{itemize}

    \end{itemize}
  \end{block}

  \vfill

  \begin{block}{Note}
    \begin{itemize}
    \item Outside of critical systems, there's no way to prove that
      all applications are secure
    \item First glimpse at \emph{Layered Security}/\emph{Defence-in-Depth}
    \end{itemize}
  \end{block}
\end{frame}




\begin{frame}
  \frametitle{Good AC systems}
  \framesubtitle{NEAT Security}

  \begin{itemize}[<+->]
  \item \textcolor{red}{N}on-bypassable: impossible to act outside of
    the bounds of the AC system
  \item \textcolor{red}{E}valuatable: possible communications can be
    assessed before deployment
  \item \textcolor{red}{A}lways invoked: no future rights, the AC
    system must always be queried
  \item \textcolor{red}{T}amperproof: the AC system cannot be changed
    while the system is running
  \end{itemize}
  
  \pause

  \begin{itemize}
  \item These concerns are shared with \emph{safe} systems
  \item Possible on systems dedicated to a task, not in general
    (\textit{e.g.}, no new file and no new user)
  \item The CS Security dilemna:
    \begin{itemize}
    \item we know how to secure a system
    \item we know how to have a usable system
    \item we know how to build a cheap system
    \end{itemize}
    but no one has ever (or ever hopes to have) more than 2 out of 3
  \end{itemize}

\end{frame}



%\begin{frame}
%  \frametitle{Disgression cryptographique}


%  \begin{itemize}[<+->]
%  \item Lors du cours de math{\'e}matiques, on a vu des primitives IND-CCA2
%  \item 
%    \begin{block}{Rappel:}
%      On autorise l'attaquant {\`a} conna{\^\i}tre le texte clair d'autant de
%      chiffr{\'e}s qu'il veut, sauf d'un en particulier. L'attaquant ne
%      doit pas pouvoir obtenir un seul bit d'information sur le clair
%      du chiffr{\'e} sp{\'e}cial
%    \end{block}
%  \item Pire cas: l'attaquant peut d{\'e}chiffrer tous les autres messages
%    \begin{center}
%      \emph{Environnement d'ex{\'e}cution d{\'e}grad{\'e}}
%    \end{center}
%  \item Attaque faible: conna{\^\i}tre un seul bit d'information
%    \begin{center}
%      Interaction minimale entre l'attaquant et le texte clair
%    \end{center}
%  \item Application du principe de s{\'e}paration:
%    \begin{itemize}
%    \item Attaquant $=$ une application
%    \item Texte clair $=$ une application
%    \item On ne veut pas qu'il y ait d'interaction entre l'attaquant
%      et le texte clair quelque soit l'environnement d'ex{\'e}cution ( $=$
%      les autres chiffr{\'e}s)
%    \end{itemize}
%  \end{itemize}
%\end{frame}



\end{reveals}
\end{document}
