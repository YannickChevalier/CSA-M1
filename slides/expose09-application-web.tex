\Titre{Sécurité du Navigateur}


  
\usepackage{ifthen}


\def\txthl#1{ \ifthenelse{\lengthtest{#1 pt<0.5pt}}{\top}{\bot} }

\begin{document}

\begin{reveals}
	
\def\pause{}
	
\maketitle


\begin{frame}
  \frametitle{But du cours}

  \vfill

  \begin{block}{Sécurité sur Internet}
    \begin{itemize}
    \item la sécurité côté serveur ne sera pas abordée
    \item la sécurité des communications repose sur TLS
    \item ce cours: sécurité du client
    \end{itemize}
  \end{block}
  \vfill

  \begin{block}{Plus généralement}
    \begin{itemize}
    \item exemple d'analyse détaillée d'un composant
    \item avantage: tout le monde connait les navigateurs
    \end{itemize}
  \end{block}
  \vfill
\end{frame}
\section{Architecture du client}

\begin{frame}
  \frametitle{S{\'e}paration spatiale (principe)}

  \begin{block}{Principe d'un browser}
    \begin{itemize}
    \item  Il affiche un document {\`a} partir duquel il est possible de
      faire des requ{\^e}tes vers des sites Webs
    \item Ce document contient du html, avec des requ{\^e}tes form{\'e}es {\`a} partir
      des balises \texttt{a}
    \item Il contient aussi des scripts, \textit{i.e.}, des programmes
      qui peuvent g{\'e}n{\'e}rer des requ{\^e}tes avec ou sans l'intervention de
      l'utilisateur
    \end{itemize}
  \end{block}


\end{frame}

\begin{frame}
    \frametitle{S{\'e}paration spatiale (pratique)}

  \begin{block}{Sandboxing}
    Chaque fen{\^e}tre d'un browser est isol{\'e}e de son environnement:
    \begin{itemize}
    \item Politique de \emph{m{\^e}me origine} (SOP): un script sur
      une page dans un sous-r\'epertoire de
      \texttt{http://www1.example.com:80/} ne peut effectuer des
      requ{\^e}tes que vers les sous-r\'epertoires du site
      \texttt{http://www1.example.com:80/}
    \item Partitionnement spatial: les browsers assurent que les
      scripts h{\'e}berg{\'e}s par diff{\'e}rentes pages:
      \begin{enumerate}
      \item ne peuvent pas avoir de variables communes
      \item ne partagent pas de codes de fonctions (mais copies de
        code autoris{\'e}es)
      \end{enumerate}
    \end{itemize}
  \end{block}
\end{frame}



\begin{frame}
  \frametitle{Contenu d'une page Web}

  \begin{block}{Base}
    \begin{itemize}
    \item Une page Web est repr{\'e}sent{\'e}e par l'arbre des balises HTML
      qu'elle contient (DOM)
    \item Une balise sp{\'e}ciale, \texttt{script}, permet d'inclure en
      plus n'importe quel programme {\'e}crit en un langage reconnu par le
      browser
    \item Langage universellement reconnu: JavaScript (JS)
    \end{itemize}
  \end{block}

  \pause

  \begin{block}{Modification d'une page Web}
    \begin{itemize}
    \item Un script contenu dans une page peut modifier cette page
    \item Il peut ajouter, modifier, ou supprimer des balises de
      n'importe quel type
    \item Le browser r{\'e}agit {\`a} ces modifications en temps r{\'e}el
    \end{itemize}
  \end{block}
\end{frame}


\begin{frame}
  \frametitle{Communications {\`a} partir d'une page Web (1/2)}

  \begin{block}{{\'e}l{\'e}ments HTML}
    \begin{itemize}
    \item Certains {\'e}l{\'e}ments HTML effectuent naturellement des requ{\^e}tes
      GET (\texttt{img}, \texttt{script}, \texttt{a} si l'utilisateur clique)
    \item Ces requ{\^e}tes sont d{\'e}finies par une url qui est recherch{\'e}e
    \item L'usage est d'encoder des {\'e}ventuelles informations {\`a} passer
      {\`a} un serveur dans cette url
    \item Les scripts peuvent aussi envoyer des requ{\^e}tes POST
      (\textit{e.g.}, soumission d'un formulaire)
    \end{itemize}
  \end{block}


\end{frame}


\begin{frame}
  \frametitle{Communications {\`a} partir d'une page Web (2/2)}

  \begin{block}{Requ{\^e}tes {\`a} partir de JavaScript}
    \begin{itemize}
    \item (universel) On cr{\'e}e un objet \texttt{XMLHttpRequest}
    \item On indique:
      \begin{itemize}
      \item quelle m{\'e}thode HTTP on d{\'e}sire utiliser 
      \item l'url {\`a} contacter
      \item {\'e}ventuellement le contenu de la requ{\^e}te
      \item la fonction {\`a} appeler lors de la r{\'e}ception de la r{\'e}ponse
        (inutile si la requ{\^e}te est synchrone)
      \end{itemize}
    \item Et voil{\`a} !
    \end{itemize}
  \end{block}

\end{frame}

\section{Partitionnement spatial}


\begin{frame}
  \frametitle{Violations du partitionnement spatial (1/5)}

  \begin{block}{Biblioth{\`e}ques}
    \begin{itemize}
    \item Il serait tr{\`e}s g{\^e}nant de demander {\`a} l'utilisateur de
      recharger une biblioth{\`e}que tr{\`e}s r{\'e}pandue (\textit{e.g.}, JQuery)
      chaque fois qu'il va sur un nouveau site
    \item Solution:
      \begin{itemize}
      \item Utilisation du cache du browser
      \item Utilisation d'un \emph{Content Distribution Network} pour
        distribuer la version de la biblioth{\`e}que qu'on utilise
      \end{itemize}
    \item Pour cela, on a besoin d'autoriser le chargement de scripts
      sur d'autres sites Web
    \end{itemize}
  \end{block}
  
  \pause

  \important{L'importation de scripts n'est jamais soumise {\`a} la
    politique de m{\^e}me origine}

\end{frame}


\begin{frame}
  \frametitle{Violations du partionnement spatial (2/5)}

  \begin{block}{Extensions}
    \begin{itemize}
    \item Les extensions des browsers peuvent d{\'e}finir des pages
      virtuelles
    \item Ces pages sont isol{\'e}es des pages affich{\'e}es (pour le
      code, pas pour les \'el\'ements)
    \item Requ{\^e}tes Javascript et \'ev\'enements possibles entre ces
      pages et celles affich{\'e}es
    \end{itemize}
  \end{block}

  \begin{block}{Cookies/localstorage}
    Stockage d'informations dans le browser
    \begin{itemize}
    \item Pas de diff{\'e}rences pour la s{\'e}curit{\'e}
    \item D{\'e}finitions de domaine diff{\'e}rente pour les cookies du reste
    \item Le code servi par un domaine ne peut utiliser que des
      informations d{\'e}finies par ce domaine
    \end{itemize}
  \end{block}

\end{frame}


\begin{frame}
  \frametitle{Violations du partionnement spatial (3/5)}
  
  \begin{block}{La s{\'e}curit{\'e}, c'est emb{\^e}tant !}
    D'apr{\`e}s ce qui a {\'e}t{\'e} {\'e}crit, il n'est pas possible
    \begin{itemize}
    \item de faire r{\'e}f{\'e}rence {\`a} une ressource h{\'e}berg{\'e}e par
      un autre domaine
    \item d'envoyer des informations personnelles sur l'utilisateur {\`a}
      un autre site (\textit{e.g.}, \texttt{doubleclick.com})
    \end{itemize}
  \end{block}

  \pause
  \begin{block}{Cross-Origin Resource Sharing (CORS)}
    \begin{itemize}
    \item Les scripts situ{\'e}s sur la page servie par le domaine A
      peuvent effectuer des requ{\^e}tes vers le domaine B
    \item Pour avoir une r{\'e}ponse, le domaine B doit \^etre
      configur{\'e} pour accepter explicitement les requ{\^e}tes venant du
      domaine A (ou de tout domaine)
    \end{itemize}
  \end{block}
  \pause

  \important{Politique de contr{\^o}le d'acc{\`e}s {\`a} compl{\'e}ter {\'e}ventuellement
    au sein domaine B avec des informations personnelles de l'utilisateur}

\end{frame}


\begin{frame}
  \frametitle{Violations du partionnement spatial (4/5)}

  \begin{block}{Rappel}
    \begin{itemize}
    \item Les scripts peuvent {\^e}tre servis par n'importe quel site
    \item Donc on peut effectuer une requ{\^e}te vers n'importe quel site
      en utilisant une balise \texttt{script} avec une url forg{\'e}e correctement
    \end{itemize}
  \end{block}


  \pause

  \begin{block}{JSON/JSONP (utilisation légitime)}
    \begin{itemize}
    \item JSON: codage d'objets JavaScript simples en du texte
    \item JSONP: enrobage du texte pour fournir un code JS valide
      (appel d'une fonction)
    \item Utilisation:
      \begin{itemize}
      \item insertion d'une balise \texttt{script} dans la page avec une
        adresse codant les informations qu'on d{\'e}sire envoyer
      \item on laisse le navigateur \emph{{\'e}valuer} le code JS re{\c c}u en
        r{\'e}ponse de la requ{\^e}te
      \end{itemize}
    \end{itemize}
  \end{block}
\end{frame}


\begin{frame}
  \frametitle{Violation du partitionnement spatial (5/5)}

  \begin{block}{frame/iframe}
    \begin{itemize}
    \item {\'E}l{\'e}ment html permettant d'inclure un autre site
    \item Les DOM sont s{\'e}par{\'e}s correctement en fonction de la SOP
    \item \url{http://qnimate.com/same-origin-policy-in-nutshell/}
    \end{itemize}
  \end{block}

  \pause
  
  \begin{block}{Violation de partition}
    \begin{itemize}
    \item Gestion complexe de l'affichage ({\`a} quels {\'e}l{\'e}ments un pixel
      est reli{\'e})
    \item Transparence
    \item En pratique, une action de l'utilisateur peut toucher
      plusieurs {\'e}l{\'e}ments, y compris en dehors et {\`a} l'int{\'e}rieur d'une
      (i)frame
    \end{itemize}
  \end{block}

\end{frame}

\begin{frame}
  \frametitle{Cookies}

  \begin{block}{D{\'e}finition du domaine}
    \begin{itemize}
    \item Par d{\'e}faut, un cookie est li{\'e} {\`a} un domaine et un chemin dans
      ce domaine (bleu ignor\'e{}):
      $$
      \underbrace{\text{\color{blue}\tt http://}}_{\text{protocole}}
      \underbrace{\text{\color{red}\tt www.example.com/}}_{\text{domaine}}
      \underbrace{\text{\color{red}\tt dir1/dir2/}}_{\text{chemin}}
      \underbrace{\text{\color{blue}\tt toto.html}}_{\text{fichier}}
      $$
    \item Par d{\'e}faut, le protocole utilis{\'e} n'est pas pris en compte
    \item Secure Cookie: en plus, il faut utiliser le protocole HTTP
    \item Il est possible de sp{\'e}cifier un autre domaine pour le cookie
      (third-party cookies), mais bloquables par l'utilisateur
    \end{itemize}
  \end{block}

  \pause

  \important{La politique de gestion des cookies est {\'e}tablie sur des
    bases diff{\'e}rentes de celle des autres ressources, ce qui est une
    source potentielle de conflit}

\end{frame}

\begin{frame}
  \frametitle{Utilisation d'un cookie}

  \begin{block}{Positionnement du domaine}
    Une page \`a{} l'adresse \texttt{a.b/c/d/} peut cr\'eer des cookies:
    \begin{itemize}
    \item pour un suffixe du domaine (en enlevant une partie
      \texttt{a}) de son adresse
    \item pour un pr\'efixe de son r\'epertoire (en enlevant un
      sous-chemin \texttt{d})
    \item il faut au moins 2 parties dans le domaine (\texttt{example}
      et \texttt{org}), plus quelques restrictions (d\'ependantes du
      browser) pour \'eviter les gags\ldots{}
    \end{itemize}
  \end{block}

  \begin{block}{Utilisation}
    \begin{itemize}
    \item Un cookie est stock\'e dans le browser
    \item Chaque fois qu'une requ\^ete est faite vers un sous-domaine
      et/ou un sous-r\'epertoire d'un cookie, le cookie est ajout\'e dans
      le header HTTP de la requ\^ete
    \item Cons\'equence (header HTTP): taille limit\'e \`a 4ko
    \item Ajout automatique, quelque soit la page faisant la requ\^ete
    \end{itemize}
  \end{block}

\end{frame}

\begin{frame}
  \frametitle{localStore et sessionStore}

  \begin{block}{Quelques caract\'eristiques en vrac}
    \begin{itemize}
    \item Li\'e \`a une page Web ($\neq$ li\'e aux requ\^etes vers\ldots{}) en
      fonction de SOP (g\'er\'e par le navigateur)
    \item Pas de limitation de taille
    \item Programmatiquement, le code JS sur une page Web a acc\`e{}s \`a 2
      variables \emph{localStorage} et \emph{sessionStorage}, et en
      fait ce qu'il veut.
    \end{itemize}
  \end{block}
\end{frame}

\section{Attaques li{\'e}es au browser}

\begin{frame}
  \frametitle{Utilisation des \'el\'ements statiques}
  
  \begin{block}{Principe}
    \begin{itemize}
    \item L'affiche d'une page par le navigateur l'oblige \`a
      r\'ecup\'erer certains \'el\'ements (essentiellement
      \texttt{link/stylesheets}, \texttt{img}, les scripts sont
      bloquables)
    \item Pour r\'ecup\'erer ces \'el\'ements, il doit envoyer des
      requ\^etes GET contenant entre autres le domaine (SOP) de la
      page faisant la requ\^ete et les cookies
    \end{itemize}
  \end{block}

  \pause{}
  \begin{block}{Exemples:}
    \begin{itemize}
    \item Insertion d'une image dans un mail pour savoir combien de
      fois il a \'et\'e ouvert (sans cookies)
      \begin{center}
        \texttt{\textcolor{red}{malicieux ou l\'egitime ?}}
      \end{center}
    \item Dans une page, coupl\'e \`a un third-party cookie, permet de
      suivre les actions d'un utilisateur 
      \begin{center}
        \texttt{\textcolor{red}{Publicit\'es, extensions Chrome/Firefox}}
      \end{center}
    \end{itemize}
  \end{block}

\end{frame}



\begin{frame}
  \frametitle{Cross-Site Request Forgery (CSRF)}

  \begin{block}{Non-blocage des images + cookies}
    \begin{itemize}
    \item {\`A} la diff{\'e}rence des localStorage, pour lesquels cela demande
      un peu plus de travail, les cookies d'un domaine sont envoy{\'e}s
      lors de chaque requ{\^e}te vers ce domaine.
    \item La page d'un domaine A peut donc faire une requ{\^e}te vers un
      domaine B en utilisant les cookies (\textit{e.g.} identification
      de l'utilisateur) de ce domaine
    \item Si le domaine B ne v{\'e}rifie pas que la requ{\^e}te est issue
      d'une page qu'il a servie, et si cette requ{\^e}te est valide, il
      va l'accepter.
    \end{itemize}
  \end{block}

  \pause

  \begin{block}{CSRF}
    \begin{itemize}
    \item Un site malicieux utilise le fait que l'utilisateur est
      potentiellement logu{\'e} sur un autre site
    \item Il envoie des requ{\^e}tes {\`a} cet autre site qui seront
      accompagn{\'e}es des cookies de cet autre site (donc authentifi{\'e}es)
    \item Ex: faire un virement si l'autre site est celui d'une
      banque, r{\'e}cup{\'e}rer une liste de contact, etc.
    \end{itemize}
  \end{block}

\end{frame}


\begin{frame}
  \frametitle{Login CSRF}
  \framesubtitle{Attaque d{\'e}pendant du browser utilis{\'e}}

  \begin{block}{Partitionnement spatial lors de la visualisation}
    \begin{itemize}
    \item un site malicieux peut superposer une image opaque pour
      couvrir une iframe
    \item L'image contient un captcha
    \item L'iframe contient la page de login d'un site (Yahoo!,
      Google, etc.)
    \end{itemize}
  \end{block}

  \pause

  \begin{block}{Login CSRF}
    \begin{itemize}
    \item Le captcha demande {\`a} l'utilisateur de rentrer le login et le
      mot de passe de l'attaquant
    \item Une fois l'utilisateur loggu{\'e}, l'attaquant r{\'e}cup{\`e}re sur son
      compte les donn{\'e}es de navigation de l'utilisateur
    \end{itemize}
  \end{block}

\end{frame}

\begin{frame}
  \frametitle{Interception de donn{\'e}es}

  \begin{block}{Probl{\`e}me de partitionnement spatial}
    \begin{itemize}
    \item Couverture d'une iframe par un {\'e}l{\'e}ment presque transparent
    \item Les {\'e}v{\'e}nements utilisateurs sont aussi envoy{\'e}s {\`a} cet {\'e}l{\'e}ment
    \item Permet au site h{\^o}te de r{\'e}cup{\'e}rer les donn{\'e}es envoy{\'e}es au
      site h{\'e}berg{\'e} dans l'iframe
    \end{itemize}
  \end{block}

  \pause

  \begin{block}{Peu de solutions}
    \begin{itemize}
    \item Beaucoup de sites utilisent de mani{\`e}re critique le
      recouvrement par des {\'e}l{\'e}ments transparents
    \item Assurer le partitionnement spatial des pixels serait tr{\`e}s
      g{\^e}nant (popup html, listes descendantes)
    \item Pas de solutions prévues
    \end{itemize}
  \end{block}
\end{frame}



\begin{frame}
  \frametitle{JSONP}

  \begin{block}{JSONP}
    \begin{itemize}
    \item L'utilisation de JSONP n{\'e}cessite une bonne collaboration
      entre les 2 domaines
    \item Mais {\`a} cause de l'{\'e}valuation, le domaine B obtient tous les
      pouvoirs (parce qu'il peut renvoyer ce qu'il veut {\`a} la requ{\^e}te,
      ce sera {\'e}valu{\'e}) sur la page affich{\'e}e et ses donn{\'e}es
    \item Autrement dit, il faut avoir une confiance absolue dans le
      domaine B pour lui faire des requ{\^e}tes \textit{via} JSONP
    \end{itemize}
  \end{block}


\end{frame}


\section{Quelques solutions possibles}


\begin{frame}
  \frametitle{Iframes}

  \begin{gitemize}
  \item Les iframes sont des sources majeures d'ins{\'e}curit{\'e} {\`a} cause de
    l'absence de politique de contr{\^o}le d'acc{\`e}s efficace sur les
    {\'e}v{\'e}nements dans le browser
  \item Seule solution pour s{\'e}curiser un site: interdire au browser de
    l'afficher dans un iframe
  \item Mais cela interdit aussi les mashup, donc rend le site moins
    utilisable
  \end{gitemize}

  \pause
  \begin{block}{Configuration de Apache (\emph{httpd.conf}}
\begin{trivlist}
\item \tt Header always append X-Frame-Options SAMEORIGIN
\item \tt Header always append X-Frame-Options DENY
\end{trivlist}
    \begin{itemize}
    \item Dans le premier cas, les iframes sont autoris{\'e}es sur le m{\^e}me
      site
    \item Dans le second, elles sont toujours refus{\'e}es
    \item Header des r{\'e}ponses {\`a} un GET, honor{\'e} par tous
      les browsers r{\'e}cents
    \end{itemize}
  \end{block}
\end{frame}

\begin{frame}
  \frametitle{CSRF}

  \begin{block}{Base des CSRF}
    \begin{itemize}
    \item L'attaquant devine que l'utilisateur est loggu{\'e} sur un
      certain site
      \begin{center}
        on n'y peut rien s'il devine correctement
      \end{center}
    \item L'attaquant peut construire des requ{\^e}tes ayant du sens
      \begin{center}
        Par contre, on peut s'arranger pour que les requ{\^e}tes incluent
        un {\'e}l{\'e}ment al{\'e}atoire non-devinable
      \end{center}
    \end{itemize}
  \end{block}

  \pause

  \begin{block}{Emp{\^e}chement des requ{\^e}tes CSRF}
    \begin{itemize}
    \item Demander au serveur de regarder la page d'origine de la
      requ{\^e}te
      \begin{center}
        utile, mais pas suffisant surtout si le site peut {\^e}tre dans
        une iframe
      \end{center}
    \item Ajouter une valeur al{\'e}atoire aux requêtes lorsqu'un utilisateur
      est loggu{\'e}
      \begin{center}
        cette valeur sera pr\'esente dans toutes les requ\^etes issues d'une
      page Web servie, il reste \`a la v\'erifier
      \end{center}
    \end{itemize}
  \end{block}
\end{frame}

\begin{frame}
  \frametitle{Emp{\^e}chement des requ{\^e}tes CSRF}
  \framesubtitle{Sp{\'e}cifique {\`a} chaque serveur d'application}

 
    \important{Application s{\'e}curis{\'e}e par d{\'e}faut}
    
  \pause
  \begin{block}{Exemple (Rails), app/controllers/application\_controller.rb}
    \begin{trivlist}
    \item \tt protect\_from\_forgery
    \end{trivlist}
  \end{block}

    \pause

    \begin{block}{Liste d'attaques}
      \url{http://guides.rubyonrails.org/security.html}
    \end{block}

\end{frame}



\begin{frame}
  \frametitle{Fragilit\'e{} de SOP}
  
  \begin{block}{Rappels:}
    \begin{itemize}
    \item Same Origin Policy: le serveur r\'epond en indiquant les
      domaines pouvant utiliser la r\'eponse
    \item Idem pour l'affichage dans une frame
    \end{itemize}
  \end{block}

  \pause{}
  \begin{block}{La s\'ecurit\'e, c'est emb\^etant$\ldots$}
    \begin{itemize}
    \item HTML5 a ajout\'e{} SOP sur plusieurs \'el\'ements (attribut \texttt{crossorigin})
    \item La s\'ecurit\'e repose sur la collaboration entre le requ\'erant et le serveur
    \item Mais il est possible d'ins\'erer un serveur non-coop\'eratif
      entre les deux pour \'eliminer les headers intempestifs
      (\url{www.anyorigin.com}, $\ldots$), contournent ces
      d\'efenses$\ldots$
    \end{itemize}
  \end{block}

\end{frame}

\begin{frame}
  \frametitle{Extensions}

  \vfill

  \begin{block}{Résumé:}
    \begin{itemize}
    \item une extension est une page Web qui est exécutée ``en mode privilégié''
    \item la SOP ne s'applique pas aux extensions
    \item \url{http://www.scs.stanford.edu/~deian/pubs/heule:2015:the-most.pdf}
    \end{itemize}
  \end{block}

  \vfill

  \begin{block}{Exemples d'attaque par des extensions}
    \begin{itemize}
    \item AdBlock, qui modifie les pages Web et enlève du revenu publicitaire
    \item diverses extensions ajoutent une image \(1\times 1\) à
      chaque page Web pour tracer l'utilisateur
    \item CSRF après récupération des données de session
    \end{itemize}
  \end{block}

  \vfill

  \important{seules les extensions fournies par un Web Store sont
    analysées, les autres peuvent tout changer}

  \vfill

\end{frame}




\end{reveals}
\end{document}
 
